%%%%%%%%%%%%%%%%%%%%%%%%%%%%%%%%%%%%%%%%%%%%%%%%%%%%%%%%%%%%%%%%%%%%%%%%%%
%Author:																 %
%-------																 %
%Yannis Baehni at University of Zurich									 %
%baehni.yannis@uzh.ch													 %
%																		 %
%Version log:															 %
%------------															 %
%06/02/16 . Basic structure												 %
%04/08/16 . Layout changes including section, contents, abstract.		 %
%%%%%%%%%%%%%%%%%%%%%%%%%%%%%%%%%%%%%%%%%%%%%%%%%%%%%%%%%%%%%%%%%%%%%%%%%%

%Page Setup
\documentclass[
	12pt, 
	oneside, 
	a4paper,
	reqno,
	final
]{amsart}

\usepackage[
	left = 3cm, 
	right = 3cm, 
	top = 3cm, 
	bottom = 3cm
]{geometry}

%Headers and footers
\usepackage{fancyhdr}
	\pagestyle{fancy}
	%Clear fields
	\fancyhf{}
	%Header right
	\fancyhead[R]{
		\footnotesize
		Yannis B\"{a}hni\\
		\href{mailto:yannis.baehni@uzh.ch}{yannis.baehni@uzh.ch}
	}
	%Header left
	\fancyhead[L]{
		\footnotesize
		MAT721 Differentiable Manifolds\\
		Spring 2017
	}
	%Page numbering in footer
	\fancyfoot[C]{\thepage}
	%Separation line header and footer
	\renewcommand{\headrulewidth}{0.4pt}
	%\renewcommand{\footrulewidth}{0.4pt}
	
	\setlength{\headheight}{19pt} 

%Title
\usepackage[foot]{amsaddr}
\usepackage{upref}
\usepackage{amssymb}
\usepackage{mathrsfs}
\DeclareMathAlphabet{\mathscrbf}{OMS}{mdugm}{b}{n}
%\usepackage{mathptmx}
\usepackage{xspace}
\makeatletter
\def\@textbottom{\vskip \z@ \@plus 1pt}
\let\@texttop\relax
\usepackage{etoolbox}
\patchcmd{\abstract}{\scshape\abstractname}{\textbf{\abstractname}}{}{}

\usepackage[all,cmtip]{xy}

%Section, subsection and subsubsection font
%------------------------------------------
\makeatletter
	\renewcommand{\@secnumfont}{\bfseries}
	\renewcommand\section{\@startsection{section}{1}%
  	\z@{.7\linespacing\@plus\linespacing}{.5\linespacing}%
  	{\normalfont\bfseries\centering}}
	\renewcommand\subsection{\@startsection{subsection}{2}%
    	\z@{.5\linespacing\@plus.7\linespacing}{-.5em}%
    	{\normalfont\bfseries}}%
	\renewcommand\subsubsection{\@startsection{subsubsection}{3}%
    	\z@{.5\linespacing\@plus.7\linespacing}{-.5em}%
    	{\normalfont\bfseries}}%
%Formatting title of TOC
\renewcommand{\contentsnamefont}{\bfseries}
%Table of Contents
\setcounter{tocdepth}{3}

% Add bold to \section titles in ToC and remove . after numbers
\renewcommand{\tocsection}[3]{%
  \indentlabel{\@ifnotempty{#2}{\bfseries\ignorespaces#1 #2\quad}}\bfseries#3}
% Remove . after numbers in \subsection
\renewcommand{\tocsubsection}[3]{%
  \indentlabel{\@ifnotempty{#2}{\ignorespaces#1 #2\quad}}#3}
\let\tocsubsubsection\tocsubsection% Update for \subsubsection
%...

\newcommand\@dotsep{4.5}
\def\@tocline#1#2#3#4#5#6#7{\relax
  \ifnum #1>\c@tocdepth % then omit
  \else
    \par \addpenalty\@secpenalty\addvspace{#2}%
    \begingroup \hyphenpenalty\@M
    \@ifempty{#4}{%
      \@tempdima\csname r@tocindent\number#1\endcsname\relax
    }{%
      \@tempdima#4\relax
    }%
    \parindent\z@ \leftskip#3\relax \advance\leftskip\@tempdima\relax
    \rightskip\@pnumwidth plus1em \parfillskip-\@pnumwidth
    #5\leavevmode\hskip-\@tempdima{#6}\nobreak
    \leaders\hbox{$\m@th\mkern \@dotsep mu\hbox{.}\mkern \@dotsep mu$}\hfill
    \nobreak
    \hbox to\@pnumwidth{\@tocpagenum{\ifnum#1=1\bfseries\fi#7}}\par% <-- \bfseries for \section page
    \nobreak
    \endgroup
  \fi}
\AtBeginDocument{%
\expandafter\renewcommand\csname r@tocindent0\endcsname{0pt}
}
\def\l@subsection{\@tocline{2}{0pt}{2.5pc}{5pc}{}}
\def\l@subsubsection{\@tocline{2}{0pt}{4.5pc}{5pc}{}}
\makeatother

\advance\footskip0.4cm
\textheight=54pc    %a4paper
\textheight=50.5pc %letterpaper
\advance\textheight-0.4cm
\calclayout

%Font settings
%\usepackage{anyfontsize}
%Footnote settings
%\usepackage{mathptmx}
\usepackage{footmisc}
%	\renewcommand*{\thefootnote}{\fnsymbol{footnote}}
\usepackage{commath}
%Further math environments
%Further math fonts (loads amsfonts implicitely)
\usepackage{amssymb}
%Redefinition of \text
%\usepackage{amstext}
\usepackage{upref}
%Graphics
%\usepackage{graphicx}
%\usepackage{caption}
%\usepackage{subcaption}
%Frames
\usepackage{mdframed}
\allowdisplaybreaks
%\usepackage{interval}
\newcommand{\toup}{%
  \mathrel{\nonscript\mkern-1.2mu\mkern1.2mu{\uparrow}}%
}
\newcommand{\todown}{%
  \mathrel{\nonscript\mkern-1.2mu\mkern1.2mu{\downarrow}}%
}
\AtBeginDocument{\renewcommand*\d{\mathop{}\!\mathrm{d}}}
\renewcommand{\Re}{\operatorname{Re}}
\renewcommand{\Im}{\operatorname{Im}}
\DeclareMathOperator\Log{Log}
\DeclareMathOperator\Arg{Arg}
\DeclareMathOperator\sech{sech}
\DeclareMathOperator*\esssup{ess.sup}
\DeclareMathOperator\id{id}
%\usepackage{hhline}
%\usepackage{booktabs} 
%\usepackage{array}
%\usepackage{xfrac} 
%\everymath{\displaystyle}
%Enumerate
\usepackage{tikz}
\usetikzlibrary{external}
\tikzexternalize % activate!
\usetikzlibrary{patterns}
\pgfdeclarepatternformonly{adjusted lines}{\pgfqpoint{-1pt}{-1pt}}{\pgfqpoint{40pt}{40pt}}{\pgfqpoint{39pt}{39pt}}%
{
  \pgfsetlinewidth{.8pt}
  \pgfpathmoveto{\pgfqpoint{0pt}{0pt}}
  \pgfpathlineto{\pgfqpoint{39.1pt}{39.1pt}}
  \pgfusepath{stroke}
}
\usepackage{enumitem} 
%\renewcommand{\labelitemi}{$\bullet$}
%\renewcommand{\labelitemii}{$\ast$}
%\renewcommand{\labelitemiii}{$\cdot$}
%\renewcommand{\labelitemiv}{$\circ$}
%Colors
%\usepackage{color}
%\usepackage[cmtip, all]{xy}
%Theorems
\newtheoremstyle{bold}              	 %Name
  {}                                     %Space above
  {}                                     %Space below
  {\itshape}		                     %Body font
  {}                                     %Indent amount
  {\bfseries}                             %Theorem head font
  {.}                                    %Punctuation after theorem head
  { }                                    %Space after theorem head, ' ', 
  										 %	or \newline
  {\thmname{#1}\thmnumber{ #2}\thmnote{ (#3)}} 
\theoremstyle{bold}
\newtheorem*{definition*}{Definition}
\newtheorem{definition}{Definition}[section]
\newtheorem*{lemma*}{Lemma}
\newtheorem{lemma}{Lemma}[section]
\newtheorem{Proof}{Proof}[section]
\newtheorem{proposition}{Proposition}[section]
\newtheorem{properties}{Properties}[section]
\newtheorem{corollary}{Corollary}[section]
\newtheorem*{theorem*}{Theorem}
\newtheorem{theorem}{Theorem}[section]
\newtheoremstyle{solution}              	 %Name
{}                                     %Space above
{}                                     %Space below
{}		      			               %Body font
{}                                     %Indent amount
{\bfseries}                             %Theorem head font
{.}                                    %Punctuation after theorem head
{ }                                    %Space after theorem head, ' ', 
%	or \newline
{\thmname{#1}\thmnumber{ #2}\thmnote{ (#3)}} 
\theoremstyle{solution}
\newtheorem*{solution}{Lösung}
\newtheorem{remark}{Remark}[section]
\newtheorem{example}{Beispiel}[subsection]
\newtheoremstyle{exercise}              	 %Name
{}                                     %Space above
{}                                     %Space below
{\small}		      			               %Body font
{}                                     %Indent amount
{\bfseries}                             %Theorem head font
{.}                                    %Punctuation after theorem head
{ }                                    %Space after theorem head, ' ', 
%	or \newline
{\thmname{#1}\thmnumber{ #2}\thmnote{ (#3)}} 
\theoremstyle{exercise}
\newtheorem{exercise}{Aufgabe}
%German non-ASCII-Characters
%Graphics-Tool
%\usepackage{tikz}
%\usepackage{tikzscale}
%\usepackage{bbm}
%\usepackage{bera}
%Listing-Setup
%Bibliographie
\usepackage[backend=bibtex, style=alphabetic]{biblatex}
%\usepackage[babel, german = swiss]{csquotes}
\bibliography{bibliography}
%PDF-Linking
%\usepackage[hyphens]{url}
\usepackage[bookmarksopen=true,bookmarksnumbered=true]{hyperref}
%\PassOptionsToPackage{hyphens}{url}\usepackage{hyperref}
\hypersetup{
  colorlinks   = true, %Colours links instead of ugly boxes
  urlcolor     = blue, %Colour for external hyperlinks
  linkcolor    = blue, %Colour of internal links
  citecolor    = blue %Colour of citations
}
%Weierstrass-P symbol for power set
\newcommand{\powerset}{\raisebox{.15\baselineskip}{\Large\ensuremath{\wp}}}
\newcommand{\bld}[1]{\boldmath\textit{\textbf{#1}}\unboldmath}
\newcommand{\Cscr}{\mathscr{C}}
\newcommand{\op}{\mathrm{op}}
\DeclareMathOperator\End{End}
\DeclareMathOperator\Hom{Hom}
\DeclareMathOperator\Mat{Mat}

\title{Complex Manifolds}
\author{Yannis B\"ahni}
\address[Yannis B\"ahni]{Universität Zürich, R\"amistrasse 71, 8006 Zurich}
\email[Yannis B\"ahni]{\href{mailto:yannis.baehni@uzh.ch}{yannis.baehni@uzh.ch}}

\begin{document}
\maketitle
\begin{abstract}

\end{abstract}
%\thispagestyle{fancy}

\tableofcontents

\section{Functions of Several Complex Variables}
This section summarizes the fundamental properties of functions of several complex variables needed later. The results are inspired by \cite[14--30]{grauert:complex_manifolds:2010}.

\begin{definition}
Let $n \in \mathbb{Z}$, $n \geq 1$, $U \subseteq \mathbb{C}^n$ open and $a \in U$. A mapping $f : U \to \mathbb{C}$ is said to be \bld{complex differentiable at $a$} if there exists $g : U \to \mathbb{C}^n$ such that $g$ is continuous at $a$ and
\begin{equation}
f(z) = f(a) + \sum_{\nu = 1}^n(z_\nu - a_\nu)g_\nu(z)
\end{equation} 
\noindent holds for all $z \in D$. $f$ is said to be \bld{holomorphic in $D$} if it is complex differentiable at every point $a \in D$. For $m \in \mathbb{Z}$, $m \geq 1$, a mapping $f : U \to \mathbb{C}^m$ is said to be holomorphic in $D$ if each component function $f_\nu$, $\nu = 1,\dots,n$, is holomorphic in $D$.
\label{def:holomorphic}
\end{definition}

\begin{proposition}
Let $n \in \mathbb{Z}$, $n \geq 1$, $D \subseteq \mathbb{C}^n$ open, $a \in U$ and $f : D \to \mathbb{C}$ real differentiable at $a$. Then
\begin{equation}
\frac{\partial f}{\partial z_\nu}(a) = \frac{1}{2}\del[3]{\frac{\partial f}{\partial x_\nu}(a) - i \frac{\partial f}{\partial y_\nu}(a)} 
\end{equation}
\noindent and
\begin{equation}
\frac{\partial f}{\partial \overline{z}_\nu}(a) = \frac{1}{2}\del[3]{\frac{\partial f}{\partial x_\nu}(a) + i \frac{\partial f}{\partial y_\nu}(a)}
\end{equation}  
\noindent holds for all $\nu = 1,\dots,n$.
\label{prop:Wirtinger}
\end{proposition}

\begin{theorem}[The Cauchy-Riemann Equations]
Let $n \in \mathbb{Z}$, $n \geq 1$ and $D \subseteq \mathbb{C}^n$ open. A mapping $f: D \to \mathbb{C}$ is holomorphic in $D$ if and only if it is real differentiable at every $a \in D$ and the \bld{Cauchy-Riemann equations}
\begin{equation}
\frac{\partial f}{\partial \overline{z}_\nu}(a) = 0 
\end{equation}
\noindent holds for all $a \in D$ and $\nu = 1,\dots,n$.
\label{thm:CR}
\end{theorem}

\begin{corollary}
Let $m,n \in \mathbb{Z}$, $m,n \geq 1$, $D \subseteq \mathbb{C}^n$ open and $f: D \to \mathbb{C}^m$ holomorphic in $D$. If $f = g + ih$, $g,h : D \to \mathbb{R}^m$, then 
\begin{equation}
\boxed{\frac{\partial g_\mu}{\partial x_\nu}(a) = \frac{\partial h_\mu}{\partial y_\nu}(a) \qquad \text{and} \qquad \frac{\partial h_\mu}{\partial x_\nu}(a) = -\frac{\partial g_\mu}{\partial y_\nu}(a)}
\end{equation} 
\noindent holds for any $a \in D$, $\nu = 1,\dots,n$ and $\mu = 1,\dots,m$.
\label{cor:CRreal}
\end{corollary}

\begin{proof}
Fix $\mu = 1,\dots,m$. By definition \ref{def:holomorphic} $f_\mu$ is holomorphic in $D$. Hence $f_\mu$ is real differentiable in $D$ (see \cite[27]{grauert:complex_manifolds:2010}) and theorem \ref{thm:CR} implies
\begin{equation*}
\frac{\partial f_\mu}{\partial \overline{z}_\nu}(a) = 0
\end{equation*}
\noindent for all $a \in D$ and $\nu = 1,\dots,n$. By proposition \ref{prop:Wirtinger}, this is equivalent to 
\begin{equation*}
\frac{\partial f}{\partial x_\nu}(a) + i \frac{\partial f}{\partial y_\nu}(a) = 0.
\end{equation*}
Using $f_\mu = g_\mu + i h_\mu$ and the $\mathbb{C}$-linearity of the operators $\frac{\partial}{\partial x_\nu}$ and $\frac{\partial}{\partial y_\nu}$ yields
\begin{equation*}
\frac{\partial g_\mu}{\partial x_\nu}(a) - \frac{\partial h_\mu}{\partial y_\nu}(a) + i \del[3]{ \frac{\partial h_\mu}{\partial x_\nu}(a) + \frac{\partial g_\mu}{\partial y_\nu}(a)} = 0.
\end{equation*}
\end{proof}

\section{Almost Complex Structures}
The following definition is taken from \cite[86]{cannas:symplectic_geometry:2008}.
\begin{definition}
Let $M$ be a smooth manifold. An \bld{almost complex structure on $M$} is a smooth tensor field $J \in \Gamma\del[1]{T^{(1,1)}TM}$ such that $J_p \circ J_p = -\id_{T_pM}$ holds for any $p \in M$. If $J$ is an almost complex structure on $M$, the tuple $(M,J)$ is called an \bld{almost complex manifold}.
\end{definition}

\begin{proposition}
Every almost complex manifold is of even dimension and orientable.
\label{prop:almost_comp}
\end{proposition}

\begin{proof}
Assume that $n := \dim M$ is odd. Let $p \in M$. Then by \cite[57]{lee:smooth_manifolds:2013} we have that $\dim T_p M = n$. Hence $\dim T_pM$ is odd. But by
\begin{equation*}
\del[1]{\det(J_p)}^2 = \det(J_p \circ J_p) = \det(-\id_{T_pM}) = (-1)^n \det(\id_{T_pM}) = (-1)^n
\end{equation*}
\noindent we see that $n$ must be even since $\det(J_p) \in \mathbb{R}$ and hence $\del[1]{\det(J_p)}^2 > 0$. Contradiction.\\
\end{proof}

\begin{remark}
The converse of proposition \ref{prop:almost_comp} is not true in general. One can show using results on fibre bundles and Chern classes, that if $\mathbb{S}^n$ admits an almost complex structure, then $n = 2^k - 2$ for $k \in \mathbb{Z}$, $k \geq 1$ (see \cite[219]{steenrod:topology_fibre:1951}). So for example $\mathbb{S}^4$ does not admit an almost complex structure. Actually, it can be shown that $\mathbb{S}^2$ and $\mathbb{S}^6$ are the only spheres which admit an almost complex structure (see \cite[434]{borel:steenrod:1953}).
\end{remark}

\section{Complex Manifolds}
The definition of smooth manifolds adapts smoothly to the complex case.

\begin{definition}
Let $n \in \mathbb{Z}$, $n \geq 1$. An \bld{$n$-dimensional complex manifold} is a second countable Hausdorff space $M$ equipped with a holomorphic structure, that is a holomorphic atlas $\cbr[0]{(U_\alpha, \varphi_\alpha) : \alpha \in A}$ of complex charts $(U_\alpha,\varphi_\alpha)$, such that all the transition maps are holomorphically compatible.
\end{definition}

\begin{lemma}
Let $V$ be a real vector space of dimension $n \in \mathbb{Z}$, $n \geq 1$. Then
\begin{equation}
V \otimes V^* \cong \End(V)
\end{equation}
\noindent canonically. If $(e_\nu)$ is a basis of $V$ and $(e_\nu^*)$ the corresponding basis of $V^*$, then $f \in \End(V)$ corresponds to
\begin{equation}
\sum_{\nu = 1}^n f(e_\nu) \otimes e_\nu^*.
\end{equation}
\label{lem:corr}
\end{lemma}

\begin{proof}
It is easily checked that
\begin{equation*}
\Phi:\begin{cases} 
V \times V^* \to \End(V)\\
(v,f) \mapsto \del[1]{u \mapsto f(u)v}
\end{cases}
\end{equation*}
\noindent is bilinear. Thus by the universal property of the tensor product there exists a unique mapping $\widehat{\Phi} \in \Hom(V \otimes V^*; \End(V))$ such that $\Phi = \widehat{\Phi} \circ \otimes$. It is also easily checked that $\widehat{\Phi}$ is an isomorphism.\\
Let $f \in \End(V)$. Then for any $v \in V$ we have
\begin{align*}
\widehat{\Phi}\del[3]{\sum_{\nu = 1}^n f(e_\nu) \otimes e_\nu^*}(v) &= \sum_{\nu = 1}^n \widehat{\Phi}\del[1]{f(e_\nu) \otimes e_\nu^*}(v)\\
&= \sum_{\nu = 1}^n e_\nu^*(v)f(e_\nu)\\
&= f\del[3]{\sum_{\nu = 1}^n e_\nu^*(v)e_\nu}\\
&= f(v).
\end{align*}
\end{proof}

\begin{proposition}
Any complex manifold admits a canonical almost complex structure.
\end{proposition}

\begin{proof}
First we define $J_p$ in terms of local coordinates. By lemma \ref{lem:corr} it is also enough to construct an endomorphism. Let $p \in M$. Given a chart $\del[1]{U,(x^\nu,y^\nu)}$ with $p \in U$, we define
\begin{equation*}
J_p\del[3]{\frac{\partial}{\partial x^\nu}\bigg\vert_p} := \frac{\partial}{\partial y^\nu}\bigg\vert_p \qquad \text{and} \qquad J_p\del[3]{\frac{\partial}{\partial y^\nu}\bigg\vert_p} := -\frac{\partial}{\partial x^\nu}\bigg\vert_p
\end{equation*}
\noindent for all $\nu = 1,\dots,n$. As standard linear algebra shows, there is a unique linear mapping associated with $J_p$ (see \cite[69]{hoffman:linear_algebra:1971}). Let $v := a^\nu\frac{\partial}{\partial x^\nu}\big\vert_p + b^\nu\frac{\partial}{\partial y^\nu}\big\vert_p \in T_p M$. Then
\begin{align*}
(J_p \circ J_p)(v) &=J_p\del[3]{a^\nu J_p\del[3]{\frac{\partial}{\partial x^\nu}\bigg\vert_p} + b^\nu J_p\del[3]{\frac{\partial}{\partial y^\nu}\bigg\vert_p}}\\
&= J_p\del[3]{a^\nu \frac{\partial}{\partial y^\nu}\bigg\vert_p - b^\nu \frac{\partial}{\partial x^\nu}\bigg\vert_p}\\
&= -a^\nu \frac{\partial}{\partial x^\nu}\bigg\vert_p - b^\nu \frac{\partial}{\partial y^\nu}\bigg\vert_p\\
&= -v
\end{align*}
\noindent and thus $J_p \circ J_p = -\id_{T_pM}$.\\
Next we have to show that above locally defined mapping is well-defined, i.e. does not depend on the choice of coordinates. Assume that $p$ lies also in the domain of the chart $\del[1]{V,(u^i,v^i)}$. By the change of coordinates formula \cite[64]{lee:smooth_manifolds:2013} we get that
\begin{equation*}
\frac{\partial}{\partial x^\nu}\bigg\vert_p = \frac{\partial u^\mu}{\partial x^\nu}(\widehat{p})\frac{\partial}{\partial u^\mu}\bigg\vert_p + \frac{\partial v^\mu}{\partial x^\nu}(\widehat{p})\frac{\partial}{\partial v^\mu}\bigg\vert_p
\end{equation*}
\noindent and
\begin{equation*}
\frac{\partial}{\partial y^\nu}\bigg\vert_p = \frac{\partial u^\mu}{\partial y^\nu}(\widehat{p})\frac{\partial}{\partial u^\mu}\bigg\vert_p + \frac{\partial v^\mu}{\partial y^\nu}(\widehat{p})\frac{\partial}{\partial v^\mu}\bigg\vert_p 
\end{equation*}
\noindent where $\widehat{p}$ denotes the coordinate representation of $p$ with respect to the coordinates $(x^\nu,y^\nu)$. Corollary \ref{cor:CRreal} implies
\begin{align*}
J_p\del[3]{\frac{\partial}{\partial x^\nu}\bigg\vert_p} &= \frac{\partial u^\mu}{\partial x^\nu}(\widehat{p})J_p\del[3]{\frac{\partial}{\partial u^\mu}\bigg\vert_p} + \frac{\partial v^\mu}{\partial x^\nu}(\widehat{p})J_p\del[3]{\frac{\partial}{\partial v^\mu}\bigg\vert_p}\\
&= \frac{\partial u^\mu}{\partial x^\nu}(\widehat{p})\frac{\partial}{\partial v^\mu}\bigg\vert_p - \frac{\partial v^\mu}{\partial x^\nu}(\widehat{p})\frac{\partial}{\partial u^\mu}\bigg\vert_p\\
&= \frac{\partial v^\mu}{\partial y^\nu}(\widehat{p})\frac{\partial}{\partial v^\mu}\bigg\vert_p + \frac{\partial u^\mu}{\partial y^\nu}(\widehat{p})\frac{\partial}{\partial u^\mu}\bigg\vert_p\\
&= \frac{\partial}{\partial y^\nu}\bigg\vert_p
\end{align*}
\noindent and
\begin{align*}
J_p\del[3]{\frac{\partial}{\partial y^\nu}\bigg\vert_p} &= \frac{\partial u^\mu}{\partial y^\nu}(\widehat{p})J_p\del[3]{\frac{\partial}{\partial u^\mu}\bigg\vert_p} + \frac{\partial v^\mu}{\partial y^\nu}(\widehat{p})J_p\del[3]{\frac{\partial}{\partial v^\mu}\bigg\vert_p}\\
&= \frac{\partial u^\mu}{\partial y^\nu}(\widehat{p})\frac{\partial}{\partial v^\mu}\bigg\vert_p - \frac{\partial v^\mu}{\partial y^\nu}(\widehat{p})\frac{\partial}{\partial u^\mu}\bigg\vert_p\\
&= -\frac{\partial v^\mu}{\partial x^\nu}(\widehat{p})\frac{\partial}{\partial v^\mu}\bigg\vert_p - \frac{\partial u^\mu}{\partial x^\nu}(\widehat{p})\frac{\partial}{\partial u^\mu}\bigg\vert_p\\
&= -\frac{\partial}{\partial x^\nu}\bigg\vert_p.
\end{align*}
Left to check is smoothness. According to lemma \ref{lem:corr} the corresponding rough tensor field is given by
\begin{equation*}
J_p\del[3]{\frac{\partial}{\partial x^\nu}\bigg\vert_p} \otimes \d x^\nu\vert_p + J_p\del[3]{\frac{\partial}{\partial y^\nu}\bigg\vert_p} \otimes \d y^\nu\vert_p = \frac{\partial}{\partial y^\nu}\bigg\vert_p \otimes \d x^\nu\vert_p - \frac{\partial}{\partial x^\nu}\bigg\vert_p \otimes \d y^\nu\vert_p
\end{equation*}
\noindent for any $p \in U$. Thus the smoothness criteria for tensor fields \cite[317--318]{lee:smooth_manifolds:2013} together with \cite[36]{lee:smooth_manifolds:2013} yields that $J \in \Gamma\del[1]{T^{(1,1)}TM}$.
\end{proof}
\printbibliography
\end{document}
