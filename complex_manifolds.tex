%%%%%%%%%%%%%%%%%%%%%%%%%%%%%%%%%%%%%%%%%%%%%%%%%%%%%%%%%%%%%%%%%%%%%%%%%%
%Author:																 %
%-------																 %
%Yannis Baehni at University of Zurich									 %
%baehni.yannis@uzh.ch													 %
%																		 %
%Version log:															 %
%------------															 %
%06/02/16 . Basic structure												 %
%04/08/16 . Layout changes including section, contents, abstract.		 %
%%%%%%%%%%%%%%%%%%%%%%%%%%%%%%%%%%%%%%%%%%%%%%%%%%%%%%%%%%%%%%%%%%%%%%%%%%

%Page Setup
\documentclass[
	12pt, 
	oneside, 
	a4paper,
	reqno,
	final
]{amsart}

\usepackage[
	left = 3cm, 
	right = 3cm, 
	top = 3cm, 
	bottom = 3cm
]{geometry}

%Headers and footers
\usepackage{fancyhdr}
	\pagestyle{fancy}
	%Clear fields
	\fancyhf{}
	%Header right
	\fancyhead[R]{
		\footnotesize
		Yannis B\"{a}hni\\
		\href{mailto:yannis.baehni@uzh.ch}{yannis.baehni@uzh.ch}
	}
	%Header left
	\fancyhead[L]{
		\footnotesize
		MAT721 Differentiable Manifolds\\
		Spring 2017
	}
	%Page numbering in footer
	\fancyfoot[C]{\thepage}
	%Separation line header and footer
	\renewcommand{\headrulewidth}{0.4pt}
	%\renewcommand{\footrulewidth}{0.4pt}
	
	\setlength{\headheight}{19pt} 

%Title
\usepackage[foot]{amsaddr}
\usepackage{upref}
\usepackage{amssymb}
\usepackage{mathrsfs}
\DeclareMathAlphabet{\mathscrbf}{OMS}{mdugm}{b}{n}
%\usepackage{mathptmx}
\usepackage{xspace}
\makeatletter
\def\@textbottom{\vskip \z@ \@plus 1pt}
\let\@texttop\relax
\usepackage{etoolbox}
\patchcmd{\abstract}{\scshape\abstractname}{\textbf{\abstractname}}{}{}

\usepackage[all,cmtip]{xy}

%Section, subsection and subsubsection font
%------------------------------------------
\makeatletter
	\renewcommand{\@secnumfont}{\bfseries}
	\renewcommand\section{\@startsection{section}{1}%
  	\z@{.7\linespacing\@plus\linespacing}{.5\linespacing}%
  	{\normalfont\bfseries\centering}}
	\renewcommand\subsection{\@startsection{subsection}{2}%
    	\z@{.5\linespacing\@plus.7\linespacing}{-.5em}%
    	{\normalfont\bfseries}}%
	\renewcommand\subsubsection{\@startsection{subsubsection}{3}%
    	\z@{.5\linespacing\@plus.7\linespacing}{-.5em}%
    	{\normalfont\bfseries}}%
%Formatting title of TOC
\renewcommand{\contentsnamefont}{\bfseries}
%Table of Contents
\setcounter{tocdepth}{3}

% Add bold to \section titles in ToC and remove . after numbers
\renewcommand{\tocsection}[3]{%
  \indentlabel{\@ifnotempty{#2}{\bfseries\ignorespaces#1 #2\quad}}\bfseries#3}
% Remove . after numbers in \subsection
\renewcommand{\tocsubsection}[3]{%
  \indentlabel{\@ifnotempty{#2}{\ignorespaces#1 #2\quad}}#3}
\let\tocsubsubsection\tocsubsection% Update for \subsubsection
%...

\newcommand\@dotsep{4.5}
\def\@tocline#1#2#3#4#5#6#7{\relax
  \ifnum #1>\c@tocdepth % then omit
  \else
    \par \addpenalty\@secpenalty\addvspace{#2}%
    \begingroup \hyphenpenalty\@M
    \@ifempty{#4}{%
      \@tempdima\csname r@tocindent\number#1\endcsname\relax
    }{%
      \@tempdima#4\relax
    }%
    \parindent\z@ \leftskip#3\relax \advance\leftskip\@tempdima\relax
    \rightskip\@pnumwidth plus1em \parfillskip-\@pnumwidth
    #5\leavevmode\hskip-\@tempdima{#6}\nobreak
    \leaders\hbox{$\m@th\mkern \@dotsep mu\hbox{.}\mkern \@dotsep mu$}\hfill
    \nobreak
    \hbox to\@pnumwidth{\@tocpagenum{\ifnum#1=1\bfseries\fi#7}}\par% <-- \bfseries for \section page
    \nobreak
    \endgroup
  \fi}
\AtBeginDocument{%
\expandafter\renewcommand\csname r@tocindent0\endcsname{0pt}
}
\def\l@subsection{\@tocline{2}{0pt}{2.5pc}{5pc}{}}
\def\l@subsubsection{\@tocline{2}{0pt}{4.5pc}{5pc}{}}
\makeatother

\advance\footskip0.4cm
\textheight=54pc    %a4paper
\textheight=50.5pc %letterpaper
\advance\textheight-0.4cm
\calclayout

%Font settings
%\usepackage{anyfontsize}
%Footnote settings
%\usepackage{mathptmx}
\usepackage{footmisc}
%	\renewcommand*{\thefootnote}{\fnsymbol{footnote}}
\usepackage{commath}
%Further math environments
%Further math fonts (loads amsfonts implicitely)
\usepackage{amssymb}
%Redefinition of \text
%\usepackage{amstext}
\usepackage{upref}
%Graphics
%\usepackage{graphicx}
%\usepackage{caption}
%\usepackage{subcaption}
%Frames
\usepackage{mdframed}
\allowdisplaybreaks
%\usepackage{interval}
\newcommand{\toup}{%
  \mathrel{\nonscript\mkern-1.2mu\mkern1.2mu{\uparrow}}%
}
\newcommand{\todown}{%
  \mathrel{\nonscript\mkern-1.2mu\mkern1.2mu{\downarrow}}%
}
\AtBeginDocument{\renewcommand*\d{\mathop{}\!\mathrm{d}}}
\renewcommand{\Re}{\operatorname{Re}}
\renewcommand{\Im}{\operatorname{Im}}
\DeclareMathOperator\Log{Log}
\DeclareMathOperator\Arg{Arg}
\DeclareMathOperator\sech{sech}
\DeclareMathOperator*\esssup{ess.sup}
\DeclareMathOperator\id{id}
%\usepackage{hhline}
%\usepackage{booktabs} 
%\usepackage{array}
%\usepackage{xfrac} 
%\everymath{\displaystyle}
%Enumerate
\usepackage{tikz}
\usetikzlibrary{external}
\tikzexternalize % activate!
\usetikzlibrary{patterns}
\pgfdeclarepatternformonly{adjusted lines}{\pgfqpoint{-1pt}{-1pt}}{\pgfqpoint{40pt}{40pt}}{\pgfqpoint{39pt}{39pt}}%
{
  \pgfsetlinewidth{.8pt}
  \pgfpathmoveto{\pgfqpoint{0pt}{0pt}}
  \pgfpathlineto{\pgfqpoint{39.1pt}{39.1pt}}
  \pgfusepath{stroke}
}
\usepackage{enumitem} 
%\renewcommand{\labelitemi}{$\bullet$}
%\renewcommand{\labelitemii}{$\ast$}
%\renewcommand{\labelitemiii}{$\cdot$}
%\renewcommand{\labelitemiv}{$\circ$}
%Colors
%\usepackage{color}
%\usepackage[cmtip, all]{xy}
%Theorems
\newtheoremstyle{bold}              	 %Name
  {}                                     %Space above
  {}                                     %Space below
  {\itshape}		                     %Body font
  {}                                     %Indent amount
  {\bfseries}                             %Theorem head font
  {.}                                    %Punctuation after theorem head
  { }                                    %Space after theorem head, ' ', 
  										 %	or \newline
  {\thmname{#1}\thmnumber{ #2}\thmnote{ (#3)}} 
\theoremstyle{bold}
\newtheorem*{definition*}{Definition}
\newtheorem{definition}{Definition}[section]
\newtheorem*{lemma*}{Lemma}
\newtheorem{lemma}{Lemma}[section]
\newtheorem{Proof}{Proof}[section]
\newtheorem{proposition}{Proposition}[section]
\newtheorem{properties}{Properties}[section]
\newtheorem{corollary}{Corollary}[section]
\newtheorem*{theorem*}{Theorem}
\newtheorem{theorem}{Theorem}[section]
\newtheoremstyle{solution}              	 %Name
{}                                     %Space above
{}                                     %Space below
{}		      			               %Body font
{}                                     %Indent amount
{\bfseries}                             %Theorem head font
{.}                                    %Punctuation after theorem head
{ }                                    %Space after theorem head, ' ', 
%	or \newline
{\thmname{#1}\thmnumber{ #2}\thmnote{ (#3)}} 
\theoremstyle{solution}
\newtheorem*{solution}{Lösung}
\newtheorem{remark}{Remark}[section]
\newtheorem{example}{Beispiel}[subsection]
\newtheoremstyle{exercise}              	 %Name
{}                                     %Space above
{}                                     %Space below
{\small}		      			               %Body font
{}                                     %Indent amount
{\bfseries}                             %Theorem head font
{.}                                    %Punctuation after theorem head
{ }                                    %Space after theorem head, ' ', 
%	or \newline
{\thmname{#1}\thmnumber{ #2}\thmnote{ (#3)}} 
\theoremstyle{exercise}
\newtheorem{exercise}{Aufgabe}
%German non-ASCII-Characters
%Graphics-Tool
%\usepackage{tikz}
%\usepackage{tikzscale}
%\usepackage{bbm}
%\usepackage{bera}
%Listing-Setup
%Bibliographie
\usepackage[backend=bibtex, style=alphabetic]{biblatex}
%\usepackage[babel, german = swiss]{csquotes}
\bibliography{bibliography}
%PDF-Linking
%\usepackage[hyphens]{url}
\usepackage[bookmarksopen=true,bookmarksnumbered=true]{hyperref}
%\PassOptionsToPackage{hyphens}{url}\usepackage{hyperref}
\hypersetup{
  colorlinks   = true, %Colours links instead of ugly boxes
  urlcolor     = blue, %Colour for external hyperlinks
  linkcolor    = blue, %Colour of internal links
  citecolor    = blue %Colour of citations
}
%Weierstrass-P symbol for power set
\newcommand{\powerset}{\raisebox{.15\baselineskip}{\Large\ensuremath{\wp}}}
\newcommand{\bld}[1]{\boldmath\textit{\textbf{#1}}\unboldmath}
\newcommand{\Cscr}{\mathscr{C}}
\newcommand{\op}{\mathrm{op}}
\DeclareMathOperator\End{End}
\DeclareMathOperator\Hom{Hom}
\DeclareMathOperator\Mat{Mat}
\DeclareMathOperator\GL{GL}
\DeclareMathOperator\supp{supp}
\newcommand{\Xfrak}{\mathfrak{X}}
\newcommand{\Ascr}{\mathscr{A}}
\newcommand{\Lup}{\mathrm{L}}

\title{Complex Manifolds}
\author{Yannis B\"ahni}
\address[Yannis B\"ahni]{University of Zurich, R\"amistrasse 71, 8006 Zurich}
\email[Yannis B\"ahni]{\href{mailto:yannis.baehni@uzh.ch}{yannis.baehni@uzh.ch}}

\begin{document}
\maketitle
\begin{abstract}
Goal of this paper is to give an overview of the basic definitions of complex and K\"ahler manifolds together with the most important properties. The main theorem will be the \emph{Newlander-Nirenberg Theorem} which gives a criterion under which an almost complex manifold is a complex one. The key role will be played by a certain tensor field, the so called \emph{Nijenhuis tensor}.
\end{abstract}
%\thispagestyle{fancy}

\tableofcontents

\section{Complex Structures on Vector Spaces}
In what follows, let $n \in \mathbb{Z}$, $n \geq 1$. Consider an $n$-dimensional complex vector space and let $J \in \End_\mathbb{C}(V)$ be defined by $J(v) := iv$. Then clearly $J \circ J = -\id_V$. Since every $n$-dimensional complex vector space can be seen as a $2n$-dimensional real vector space in a natural way, i.e. if $(e_\nu)$ is a basis for the complex vector space $V$, then $(e_\nu,ie_\nu)$ is a basis for the real vector space $V$, the mapping $J$ induces an $\mathbb{R}$-endomorphism $\widetilde{J}$ on the real vector space $V$ simply by defining $\widetilde{J}(e_\nu) := J(e_\nu) = ie_\nu$ and $\widetilde{J}(ie_\nu) := J(ie_\nu) = -e_\nu$ for all $\nu = 1,\dots,n$.\\
Conversly, let $V$ be an $n$-dimensional real vector space with $J \in \End_\mathbb{R}(V)$ such that $J \circ J = -\id_V$. One can show, that 
\begin{equation}
zv := xv + yJ(v)
\label{eq:complex}
\end{equation}
\noindent for $z := x + iy \in \mathbb{C}$ and $v \in V$ makes $V$ into a complex vector space. This motivates the following definition.
\begin{definition}
Let $n \in \mathbb{Z}$, $n \geq 1$ and $V$ be an $n$-dimensional real vector space. A \bld{complex structure on $V$} is a $\mathbb{R}$-linear mapping $J: V \to V$ such that $J \circ J = -\id_V$. If $J$ is a complex structure on $V$, the tuple $(V,J)$ is called a \bld{complex vector space}.
\end{definition}

\begin{lemma}
Let $(V,J)$ be a complex vector space. Then $\dim V$ is even.
\label{lem:complex_vec}
\end{lemma}

\begin{proof}
That $\dim V$ must be even follows directly from
\begin{equation*}
\del[1]{\det(J)}^2 = \det(J \circ J) = \det(-\id_V) = (-1)^{\dim V} \det(\id_V) = (-1)^{\dim V}
\end{equation*} 
\noindent since $\det(J) \in \mathbb{R}$.
\end{proof}

\section{Almost Complex Structures}
If $M$ is a smooth manifold, then $T_pM$ is a finite dimensional real vector space. Hence we can generalize the definitions and results of the previous section to manifolds. The following definition is taken from \cite[86]{cannas:symplectic_geometry:2008}.
\begin{definition}
Let $M$ be a smooth manifold. An \bld{almost complex structure on $M$} is a smooth tensor field $J \in \Gamma\del[1]{T^{(1,1)}TM}$ such that $J_p \circ J_p = -\id_{T_pM}$ holds for every $p \in M$. If $J$ is an almost complex structure on $M$, the tuple $(M,J)$ is called an \bld{almost complex manifold}.
\end{definition}

\begin{proposition}
Every almost complex manifold $(M,J)$ is of even dimension and orientable.
\label{prop:almost_comp}
\end{proposition}

\begin{proof}
Assume that $\dim M$ is odd. Let $p \in M$. Then by \cite[57]{lee:smooth_manifolds:2013} we have that $\dim T_p M = \dim M$. Hence $\dim T_pM$ is odd. But by lemma \ref{lem:complex_vec}, $\dim T_p M$ must be even since $(T_pM,J_p)$ is a complex vector space. Contradiction.\\
Since $M$ is a smooth manifold, there exists a Riemannian metric $g$ on $M$ (see \cite[329]{lee:smooth_manifolds:2013}). Define 
\begin{equation*}
\widetilde{g}(X,Y) := g(X,Y) + g(JX,JY) \in \Gamma(T^{(0,2)}TM)
\end{equation*}
\noindent for all $X,Y \in \mathfrak{X}(M)$. This is possible due to the tensor characterization lemma \ref{thm:tensor_char}. Then
\begin{equation*}
\widetilde{g}(JX,JY) = g(JX,JY) + g(-X,-Y) = g(JX,JY) + g(X,Y) = \widetilde{g}(X,Y)
\end{equation*}
\noindent by the bilinearity of $g$. Furthermore, clearly $\widetilde{g}$ is positive definite and symmetric, thus a Riemannian metric on $M$. Define
\begin{equation*}
\omega(X,Y) := \widetilde{g}(X,JY).
\end{equation*}
Then by
\begin{equation*}
\omega(Y,X) = \widetilde{g}(Y,JX) = \widetilde{g}(JX,Y) = \widetilde{g}(-X,JY) = -\omega(X,Y)
\end{equation*}
\noindent we see that $\omega$ is skew-symmetric. Hence $\omega \in \Omega^2(M)$. Let $p \in M$ and $u \in T_pM \setminus \cbr{0}$. Then also $-J_p(u) \neq 0$ since $J_p$ is invertible by $\det J_p = 1$. Furthermore, by \cite[177]{lee:smooth_manifolds:2013}, there exist $X,Y \in \Xfrak(M)$, such that $X_p = u$ and $Y_p = -J_p(u)$. Hence
\begin{align*}
\omega_p\del[1]{u,-J_p(u)} &= \omega_p(X_p,Y_p)\\
&= \omega(X,Y)(p)\\
&= \widetilde{g}(X,JY)(p)\\
&= \widetilde{g}_p\del[1]{X_p,(JY)_p}\\
&= \widetilde{g}_p\del[1]{u,J_p(Y_p)}\\
&= \widetilde{g}_p\del[1]{u,-(J_p \circ J_p)(u)}\\
&= \widetilde{g}_p(u,u)\\
&\neq 0
\end{align*}
\noindent and by \cite[565]{lee:smooth_manifolds:2013} we get that $\omega$ is nondegenrate. Let $\dim M = 2n$. By \cite[567]{lee:smooth_manifolds:2013} this implies that $\omega_p \wedge \dots \wedge \omega_p$ is nonzero for each $p \in M$. Hence $\omega \wedge \dots \wedge \omega$ is a nonvanishing top form on $M$. Since any nonvanishing top form determines an orientation (see \cite[381]{lee:smooth_manifolds:2013}), we have that $M$ is orientable.
\end{proof}

\begin{remark}
The converse of proposition \ref{prop:almost_comp} is not true in general. One can show using results on fibre bundles and Chern classes, that if $\mathbb{S}^n$ admits an almost complex structure, then $n = 2^k - 2$ for $k \in \mathbb{Z}$, $k \geq 1$ (see \cite[219]{steenrod:topology_fibre:1951}). So for example $\mathbb{S}^4$ does not admit an almost complex structure. Actually, it can be shown that $\mathbb{S}^2$ and $\mathbb{S}^6$ are the only spheres which admit an almost complex structure (see \cite[434]{borel:steenrod:1953}).
\end{remark}

\section{Complex Manifolds}
The definition of smooth manifolds adapts smoothly to the complex case.

\begin{definition}
Let $n \in \mathbb{Z}$, $n \geq 1$. An \bld{$n$-dimensional complex manifold} is a second countable Hausdorff space $M$ equipped with a holomorphic structure, that is a maximal holomorphic atlas of complex charts, such that all the transition maps are holomorphic.
\end{definition}

\begin{examples}[Complex Manifolds]
~
\begin{enumerate}[label = \arabic*.]
\item The complex $n$-space $\mathbb{C}^n$ is an $n$-dimensional complex manifold. 
\item Let $\cbr[0]{\omega_1,\dots,\omega_{2n}}$ be a real basis of $\mathbb{C}^n$ and define
\begin{equation}
G := \mathbb{Z}\omega_1 + \dots + \mathbb{Z}\omega_{2n}.
\end{equation}
Then the discrete group $G$ acts freely and properly discontinuously on $\mathbb{C}^n$ by translation. Thus $\mathbb{T}^n := \mathbb{C}^n/G$ is an $n$-dimensional complex manifold, called a \bld{complex torus} (see \cite[206--207]{grauert:complex_manifolds:2010}). 
\item The quotient $(\mathbb{C}^{n + 1} \setminus \cbr[0]{0})/\mathbb{C}^\times$ is an $n$-dimensional complex manifold, called the \bld{complex projective space} (see \cite[208--210]{grauert:complex_manifolds:2010}).
\end{enumerate}
\end{examples}

\begin{lemma}
Let $n,k \in \mathbb{Z}$, $n,k \geq 1$. Let $V$ be an $n$-dimensional real vector space. Then
\begin{equation}
V \otimes \underbrace{V^* \otimes \dots \otimes V^*}_{k} \cong \Lup(\underbrace{V,\dots,V}_{k};V)
\end{equation}
\noindent canonically. If $(e_\nu)$ is a basis of $V$ and $(e_\nu^*)$ the corresponding basis of $V^*$, then $f \in \End(V)$ corresponds to
\begin{equation}
\sum_{\nu = 1}^n f(e_\nu) \otimes e_\nu^*.
\end{equation}
\label{lem:corr}
\end{lemma}

\begin{proof}
It is easily checked that
\begin{equation*}
\Psi:\begin{cases} 
V \times V^* \times \dots \times V^* \to \Lup(V,\dots,V;V)\\
(v,f_1,\dots,f_k) \mapsto \del[1]{(v_1,\dots,v_k) \mapsto f_1(v_1)\cdots f_k(v_k)v}
\end{cases}
\end{equation*}
\noindent is multilinear. Thus by the universal property of the tensor product there exists a unique linear mapping $\widetilde{\Psi} \in \Hom_\mathbb{R}\del[1]{T^{(1,k)}(V); \Lup(V,\dots,V;V)}$ such that $\Psi = \widetilde{\Psi} \circ \otimes$. It is also easily checked that $\widetilde{\Psi}$ is an isomorphism. Let $f \in \End(V)$. Then for any $v \in V$ we have
\begin{align*}
\widetilde{\Psi}\del[3]{\sum_{\nu = 1}^n f(e_\nu) \otimes e_\nu^*}(v) &= \sum_{\nu = 1}^n \widetilde{\Psi}\del[1]{f(e_\nu) \otimes e_\nu^*}(v)\\
&= \sum_{\nu = 1}^n e_\nu^*(v)f(e_\nu)\\
&= f\del[3]{\sum_{\nu = 1}^n e_\nu^*(v)e_\nu}\\
&= f(v).
\end{align*}
\end{proof}

\begin{proposition}
Any complex manifold admits a canonical almost complex structure.
\label{prop:complex_is_almost}
\end{proposition}

\begin{proof}
Fix a complex manifold $M$. We define $J$ in terms of local coordinates. Let $\del[1]{U,(x^\nu,y^\nu)}$ be a chart. By lemma \ref{lem:corr} it is also enough to construct an endomorphism $J_p$ for every $p \in U$. We define
\begin{equation*}
J_p\del[3]{\frac{\partial}{\partial x^\nu}\bigg\vert_p} := \frac{\partial}{\partial y^\nu}\bigg\vert_p \qquad \text{and} \qquad J_p\del[3]{\frac{\partial}{\partial y^\nu}\bigg\vert_p} := -\frac{\partial}{\partial x^\nu}\bigg\vert_p
\end{equation*}
\noindent for all $\nu = 1,\dots,n$. As standard linear algebra shows, there is a unique linear mapping associated with $J_p$ (see \cite[69]{hoffman:linear_algebra:1971}). Let $v := a^\nu\frac{\partial}{\partial x^\nu}\big\vert_p + b^\nu\frac{\partial}{\partial y^\nu}\big\vert_p \in T_p M$. Then
\begin{align*}
(J_p \circ J_p)(v) &=J_p\del[3]{a^\nu J_p\del[3]{\frac{\partial}{\partial x^\nu}\bigg\vert_p} + b^\nu J_p\del[3]{\frac{\partial}{\partial y^\nu}\bigg\vert_p}}\\
&= J_p\del[3]{a^\nu \frac{\partial}{\partial y^\nu}\bigg\vert_p - b^\nu \frac{\partial}{\partial x^\nu}\bigg\vert_p}\\
&= -a^\nu \frac{\partial}{\partial x^\nu}\bigg\vert_p - b^\nu \frac{\partial}{\partial y^\nu}\bigg\vert_p\\
&= -v
\end{align*}
\noindent and thus $J_p \circ J_p = -\id_{T_pM}$.\\
Next we have to show that above locally defined mapping is well-defined, i.e. does not depend on the choice of coordinates. Assume that $p \in U \cap V$ for another chart $\del[1]{V,(u^i,v^i)}$. By the change of coordinates formula \cite[64]{lee:smooth_manifolds:2013} we get that
\begin{equation*}
\frac{\partial}{\partial x^\nu}\bigg\vert_p = \frac{\partial u^\mu}{\partial x^\nu}(\widehat{p})\frac{\partial}{\partial u^\mu}\bigg\vert_p + \frac{\partial v^\mu}{\partial x^\nu}(\widehat{p})\frac{\partial}{\partial v^\mu}\bigg\vert_p
\end{equation*}
\noindent and
\begin{equation*}
\frac{\partial}{\partial y^\nu}\bigg\vert_p = \frac{\partial u^\mu}{\partial y^\nu}(\widehat{p})\frac{\partial}{\partial u^\mu}\bigg\vert_p + \frac{\partial v^\mu}{\partial y^\nu}(\widehat{p})\frac{\partial}{\partial v^\mu}\bigg\vert_p 
\end{equation*}
\noindent where $\widehat{p}$ denotes the coordinate representation of $p$ with respect to the coordinates $(x^\nu,y^\nu)$. Corollary \ref{cor:CRreal} implies
\begin{align*}
J_p\del[3]{\frac{\partial}{\partial x^\nu}\bigg\vert_p} &= \frac{\partial u^\mu}{\partial x^\nu}(\widehat{p})J_p\del[3]{\frac{\partial}{\partial u^\mu}\bigg\vert_p} + \frac{\partial v^\mu}{\partial x^\nu}(\widehat{p})J_p\del[3]{\frac{\partial}{\partial v^\mu}\bigg\vert_p}\\
&= \frac{\partial u^\mu}{\partial x^\nu}(\widehat{p})\frac{\partial}{\partial v^\mu}\bigg\vert_p - \frac{\partial v^\mu}{\partial x^\nu}(\widehat{p})\frac{\partial}{\partial u^\mu}\bigg\vert_p\\
&= \frac{\partial v^\mu}{\partial y^\nu}(\widehat{p})\frac{\partial}{\partial v^\mu}\bigg\vert_p + \frac{\partial u^\mu}{\partial y^\nu}(\widehat{p})\frac{\partial}{\partial u^\mu}\bigg\vert_p\\
&= \frac{\partial}{\partial y^\nu}\bigg\vert_p
\end{align*}
\noindent and
\begin{align*}
J_p\del[3]{\frac{\partial}{\partial y^\nu}\bigg\vert_p} &= \frac{\partial u^\mu}{\partial y^\nu}(\widehat{p})J_p\del[3]{\frac{\partial}{\partial u^\mu}\bigg\vert_p} + \frac{\partial v^\mu}{\partial y^\nu}(\widehat{p})J_p\del[3]{\frac{\partial}{\partial v^\mu}\bigg\vert_p}\\
&= \frac{\partial u^\mu}{\partial y^\nu}(\widehat{p})\frac{\partial}{\partial v^\mu}\bigg\vert_p - \frac{\partial v^\mu}{\partial y^\nu}(\widehat{p})\frac{\partial}{\partial u^\mu}\bigg\vert_p\\
&= -\frac{\partial v^\mu}{\partial x^\nu}(\widehat{p})\frac{\partial}{\partial v^\mu}\bigg\vert_p - \frac{\partial u^\mu}{\partial x^\nu}(\widehat{p})\frac{\partial}{\partial u^\mu}\bigg\vert_p\\
&= -\frac{\partial}{\partial x^\nu}\bigg\vert_p.
\end{align*}
Left to check is smoothness. According to lemma \ref{lem:corr} the corresponding rough tensor field is given by
\begin{equation*}
J_p\del[3]{\frac{\partial}{\partial x^\nu}\bigg\vert_p} \otimes \d x^\nu\vert_p + J_p\del[3]{\frac{\partial}{\partial y^\nu}\bigg\vert_p} \otimes \d y^\nu\vert_p = \frac{\partial}{\partial y^\nu}\bigg\vert_p \otimes \d x^\nu\vert_p - \frac{\partial}{\partial x^\nu}\bigg\vert_p \otimes \d y^\nu\vert_p
\end{equation*}
\noindent for any $p \in U$. Thus the smoothness criteria for tensor fields \ref{prop:smoothness_tensor} together with \cite[36]{lee:smooth_manifolds:2013} yields that $J \in \Gamma\del[1]{T^{(1,1)}TM}$.
\end{proof}

A question which naturally arises by considering proposition \ref{prop:complex_is_almost} is, if the converse is also true, i.e. if every almost complex manifold is a complex manifold. This is in general not the case. Let $\mathbb{P}$ denote the naturally oriented underlying smooth manifold of the complex projective plane. Again using results about Chern numbers it can be shown that
\begin{equation}
\mathbb{P} \# (\mathbb{S}^1 \times \mathbb{S}^3) \# (\mathbb{S}^1 \times \mathbb{S}^3) \qquad \text{and} \qquad (\mathbb{S}^1 \times \mathbb{S}^3) \# (\mathbb{S}^1 \times \mathbb{S}^3) \# (\mathbb{S}^2 \times \mathbb{S}^2)
\end{equation}
\noindent have almost complex structures but no complex structure (see \cite[1627]{ven:chern:1966}). 

\section{The Nijenhuis Tensor and the Newlander-Nirenberg Theorem}
As we have seen in the last section, not every almost complex manifold is a complex manifold. Under which condition is this possible?

\begin{definition}
Let $(M,J)$ be an almost complex manifold. For $X,Y \in \mathfrak{X}(M)$ we define the \bld{Nijenhuis tensor $N$} as
\begin{equation}
N(X,Y) := \sbr[0]{JX,JY} - J\sbr[0]{X,JY} - J\sbr[0]{JX,Y} - \sbr[0]{X,Y}
\end{equation}
\noindent where $\sbr[0]{X,Y}$ denotes the usual Lie-bracket of vector fields.
\end{definition}

\begin{proposition}
Let $(M,J)$ be an almost complex manifold and $N$ be the Nijenhuis tensor. Then $N \in \Gamma\del[1]{T^{(1,2)}TM}$.
\label{prop:nijenhuis_tensor}
\end{proposition}

\begin{proof}
First of all, $N(X,Y) \in \mathfrak{X}(M)$ for all $X,Y \in \mathfrak{X}(M)$. This follows immediately by considering $J$ as a mapping $J : \mathfrak{X}(M) \to \mathfrak{X}(M)$ using the tensor characterization lemma \ref{thm:tensor_char}, the fact that the Lie Bracket of two smooth vector fields is again a smooth vector field (see \cite[186]{lee:smooth_manifolds:2013}) and that $\mathfrak{X}(M)$ is a $\Cscr^\infty(M)$-module (see \cite[177]{lee:smooth_manifolds:2013}). Let $f \in \Cscr^\infty(M)$ and $X,Y,Z \in \mathfrak{X}(M)$. Then
\begin{align*}
N(fX + Y,Z) =& \sbr[0]{J(fX + Y),JZ} - J\sbr[0]{fX + Y,JZ} - J\sbr[0]{J(fX + Y),Z}\\
 &- \sbr[0]{fX + Y,Z}\\
=& \sbr[0]{fJX + JY,JZ} - J\sbr[0]{fX + Y,JZ} - J\sbr[0]{fJX + JY,Z}\\
& - \sbr[0]{fX + Y,Z}\\
=& \sbr[0]{fJX,JZ} + \sbr[0]{JY,JZ} - J\sbr[0]{fX,JZ} - J\sbr[0]{Y,JZ} - J\sbr[0]{fJX,Z}\\
& -J\sbr[0]{JY,Z} - \sbr[0]{fX,Z} - \sbr[0]{Y,Z}\\
=& f\sbr[0]{JX,JZ} - (JZf)JX + \sbr[0]{JY,JZ} - fJ\sbr[0]{X,JZ} + (JZf)JX\\
& - \sbr[0]{Y,JZ} - fJ\sbr[0]{JX,Z} + (Zf)JJX - J\sbr[0]{JY,Z} - f\sbr[0]{X,Z}\\
& + (Zf)X - \sbr[0]{Y,Z}\\
=& fN(X,Z) + N(Y,Z).
\end{align*}
\noindent by \cite[187--188]{lee:smooth_manifolds:2013}. Linearity in the second argument is shown similarly. Hence $N : \mathfrak{X}(M) \times \mathfrak{X}(M) \to \mathfrak{X}(M)$ is bilinear over $\Cscr^\infty(M)$. Again by the tensor characterization lemma \ref{thm:tensor_char} we have that $N \in \Gamma\del[1]{T^{(1,2)}TM}$.
\end{proof}

\begin{theorem}[Newlander-Nirenberg]
Let $(M,J)$ be an almost complex manifold. Then $M$ is a complex manifold, where the complex structure is so that the canonically induced almost complex structure is $J$, if and only if the Nijenhuis tensor $N$ vanishes identically.
\end{theorem}

\begin{proof}
Assume $M$ is a complex manifold. Let $(U,(x^\nu,y^\nu))$ be a chart. From proposition \ref{prop:nijenhuis_tensor} it is enough to consider the coordinate vector fields $\frac{\partial}{\partial x^\nu}$ and $\frac{\partial}{\partial y^\nu}$. But from the explicit definition of $J$  in proposition \ref{prop:complex_is_almost} and the property, that the Lie-Bracket of coordinate vector fields vanishes, together with the $\Cscr^\infty(M)$-linearity of $J$ we get that $N$ vanishes identically on each chart, and thus on $M$.\\
The other direction however is far more technical and uses results on partial differential equations. A complete proof can either be found in the original paper \cite{newlander:coordinates:1957} or in \cite[106]{cannas:symplectic_geometry:2008}, where references to more recent proofs are given.
\end{proof}

\section{K\"ahler Manifolds}
The following is inspired by \cite[146--149]{kobayashi:diff_geo_II:1996} and introduces the concepts from a complex viewpoint. This is in contrast to the symplectic approach provided for example in \cite{cannas:symplectic_geometry:2008}.
\begin{definition}
Let $(M,J)$ be an almost complex manifold. A \bld{Hermitian metric on $M$} is a Riemannian metric $g$ such that 
\begin{equation}
g(JX,JY) = g(X,Y)
\end{equation} 
\noindent holds for all $X,Y \in \Xfrak(M)$. If $g$ is a Hermitian metric on $M$, the triple $(M,J,g)$ is called an \bld{almost Hermitian manifold}.
\end{definition}

\begin{lemma}
Every almost complex manifold admits a Hermitian metric.
\end{lemma}

\begin{proof}
The existence was shown in the proof of proposition \ref{prop:almost_comp}.
\end{proof}

\begin{definition}
Let $(M,J,g)$ be an almost Hermitian manifold. The \bld{fundamental $2$-form $\Omega$} is defined to be
\begin{equation}
\Omega(X,Y) := g(X,JY)
\end{equation}
\noindent for all $X,Y \in \Xfrak(M)$.
\end{definition}

\begin{definition}
Let $(M,J,g)$ be an almost Hermitian manifold with fundamental $2$-form $\Omega$. The Hermitian metric is said to be a \bld{K\"ahler metric}, if $\d \Omega = 0$. An almost complex manifold with a K\"ahler metric is called an \bld{almost K\"ahler manifold} and a complex manifold with a K\"ahler metric is called a \bld{K\"ahler manifold}.
\end{definition}

\begin{appendix}
\section{Functions of Several Complex Variables}
This section summarizes the fundamental properties of functions of several complex variables needed later. The results are inspired by \cite[14--30]{grauert:complex_manifolds:2010}.

\begin{definition}
Let $n \in \mathbb{Z}$, $n \geq 1$, $U \subseteq \mathbb{C}^n$ open and $a \in U$. A mapping $f : U \to \mathbb{C}$ is said to be \bld{complex differentiable at $a$} if there exists $g : U \to \mathbb{C}^n$ such that $g$ is continuous at $a$ and
\begin{equation}
f(z) = f(a) + \sum_{\nu = 1}^n(z_\nu - a_\nu)g_\nu(z)
\end{equation} 
\noindent holds for all $z \in D$. $f$ is said to be \bld{holomorphic in $D$} if it is complex differentiable at every point $a \in D$. For $m \in \mathbb{Z}$, $m \geq 1$, a mapping $f : U \to \mathbb{C}^m$ is said to be holomorphic in $D$ if each component function $f_\nu$, $\nu = 1,\dots,n$, is holomorphic in $D$.
\label{def:holomorphic}
\end{definition}

\begin{proposition}
Let $n \in \mathbb{Z}$, $n \geq 1$, $D \subseteq \mathbb{C}^n$ open, $a \in U$ and $f : D \to \mathbb{C}$ real differentiable at $a$. Then
\begin{equation}
\frac{\partial f}{\partial z_\nu}(a) = \frac{1}{2}\del[3]{\frac{\partial f}{\partial x_\nu}(a) - i \frac{\partial f}{\partial y_\nu}(a)} 
\end{equation}
\noindent and
\begin{equation}
\frac{\partial f}{\partial \overline{z}_\nu}(a) = \frac{1}{2}\del[3]{\frac{\partial f}{\partial x_\nu}(a) + i \frac{\partial f}{\partial y_\nu}(a)}
\end{equation}  
\noindent holds for all $\nu = 1,\dots,n$.
\label{prop:Wirtinger}
\end{proposition}

\begin{theorem}[The Cauchy-Riemann Equations]
Let $n \in \mathbb{Z}$, $n \geq 1$ and $D \subseteq \mathbb{C}^n$ open. A mapping $f: D \to \mathbb{C}$ is holomorphic in $D$ if and only if it is real differentiable at every $a \in D$ and the \bld{Cauchy-Riemann equations}
\begin{equation}
\frac{\partial f}{\partial \overline{z}_\nu}(a) = 0 
\end{equation}
\noindent holds for all $a \in D$ and $\nu = 1,\dots,n$.
\label{thm:CR}
\end{theorem}

\begin{corollary}
Let $m,n \in \mathbb{Z}$, $m,n \geq 1$, $D \subseteq \mathbb{C}^n$ open and $f: D \to \mathbb{C}^m$ holomorphic in $D$. If $f = g + ih$, $g,h : D \to \mathbb{R}^m$, then 
\begin{equation}
\boxed{\frac{\partial g_\mu}{\partial x_\nu}(a) = \frac{\partial h_\mu}{\partial y_\nu}(a) \qquad \text{and} \qquad \frac{\partial h_\mu}{\partial x_\nu}(a) = -\frac{\partial g_\mu}{\partial y_\nu}(a)}
\end{equation} 
\noindent holds for any $a \in D$, $\nu = 1,\dots,n$ and $\mu = 1,\dots,m$.
\label{cor:CRreal}
\end{corollary}

\begin{proof}
Fix $\mu = 1,\dots,m$. By definition \ref{def:holomorphic} $f_\mu$ is holomorphic in $D$. Hence $f_\mu$ is real differentiable in $D$ (see \cite[27]{grauert:complex_manifolds:2010}) and theorem \ref{thm:CR} implies
\begin{equation*}
\frac{\partial f_\mu}{\partial \overline{z}_\nu}(a) = 0
\end{equation*}
\noindent for all $a \in D$ and $\nu = 1,\dots,n$. By proposition \ref{prop:Wirtinger}, this is equivalent to 
\begin{equation*}
\frac{\partial f}{\partial x_\nu}(a) + i \frac{\partial f}{\partial y_\nu}(a) = 0.
\end{equation*}
Using $f_\mu = g_\mu + i h_\mu$ and the $\mathbb{C}$-linearity of the operators $\frac{\partial}{\partial x_\nu}$ and $\frac{\partial}{\partial y_\nu}$ yields
\begin{equation*}
\frac{\partial g_\mu}{\partial x_\nu}(a) - \frac{\partial h_\mu}{\partial y_\nu}(a) + i \del[3]{ \frac{\partial h_\mu}{\partial x_\nu}(a) + \frac{\partial g_\mu}{\partial y_\nu}(a)} = 0.
\end{equation*}
\end{proof}
\section{Tensor Characterization Lemma}
\begin{definition}
Let $k,l \in \mathbb{Z}$, $k,l \geq 0$ and $M$ a smooth manifold. Then the \bld{bundle of mixed tensors of type $(k,l)$} is defined by
\begin{equation}
T^{(k,l)}TM := \coprod_{p \in M} T^{(k,l)}(T_pM).
\end{equation}
\end{definition}

\begin{proposition}
The bundle of mixed tensors of type $(k,l)$ has an unique natural structure as a smooth vector bundle of rank $n^{k + l}$ over $M$.
\label{prop:smooth_bundle}
\end{proposition}

\begin{proof}
For each $p \in M$ let $E_p := T^{(k,l)}(T_pM)$. By \cite[57]{lee:smooth_manifolds:2013} and \cite[313]{lee:smooth_manifolds:2013} $\dim E_p = n^{k + l}$. Furthermore, let $E := T^{(k,l)}TM$ and $\pi : E \to M$ be defined by $\pi(p,A) := p$. Let $\cbr[0]{(U_\alpha,\varphi_\alpha) : \alpha \in A}$ be an atlas for $M$. For each $\alpha \in A$ define
\begin{equation*}
\Phi_\alpha:\begin{cases}
\pi^{-1}(U_\alpha) \to U_\alpha \times \mathbb{R}^{n^{k + l}}\\
(p,A) \mapsto \del[1]{p,(A^{i_1\dots i_k}_{j_1\dots j_l})}
\end{cases}
\end{equation*}
Clearly, the inverse is given by
\begin{equation*}
\Phi^{-1}_\alpha:\begin{cases}
U_\alpha \times\mathbb{R}^{n^{k + l}} \to \pi^{-1}(U_\alpha)\\
\del[1]{p,(A^{i_1\dots i_k}_{j_1\dots j_l})} \mapsto (p,A)
\end{cases}.
\end{equation*}
Hence each $\Phi_\alpha$ is bijective. Now we have to check, that $\Phi_\alpha\vert_{E_p}$ is an isomorphism. So let $\lambda \in \mathbb{R}$ and $B \in E_p$. Then
\begin{align*}
\Phi_\alpha\vert_{E_p}(p,\lambda A + B) &= \del[1]{p,(\lambda A + B)^{i_1\dots i_k}_{j_1\dots j_l})}\\
&= \del[1]{p,\lambda (A^{i_1\dots i_k}_{j_1\dots j_l}) + (B^{i_1\dots i_k}_{j_1\dots j_l})}\\
&= \lambda\Phi_\alpha\vert_{E_p}(p,A) + \Phi_\alpha\vert_{E_p}(p,B).
\end{align*}
Now let $\alpha, \beta \in A$ such that $U_\alpha \cap U_\beta \neq \varnothing$. We consider the mapping
\begin{equation*}
\Phi_\alpha \circ \Phi_\beta^{-1} : (U_\alpha \cap U_\beta) \times \mathbb{R}^{n^{k + l}} \to (U_\alpha \cap U_\beta) \times \mathbb{R}^{n^{k + l}}.
\end{equation*}
Define $\tau_{\alpha\beta}: U_\alpha \cap U_\beta \to \GL(n^{k + l},\mathbb{R})$ by
\begin{equation*}
\tau_{\alpha\beta} := (\delta^i_j).
\end{equation*}
Then we have that
\begin{equation*}
(\Phi_\alpha \circ \Phi_\beta^{-1})\del[1]{p,(A^{i_1\dots i_k}_{j_1\dots j_l})} = \del[1]{p,(A^{i_1\dots i_k}_{j_1\dots j_l})} = \del[1]{p,\tau_{\alpha\beta}(p)(A^{i_1\dots i_k}_{j_1\dots j_l})}.
\end{equation*}
Since $\tau_{\alpha\beta}$ is constant, it is smooth (see \cite[36]{lee:smooth_manifolds:2013}). Hence we can apply the vector bundle chart lemma \cite[253]{lee:smooth_manifolds:2013} and the result follows.
\end{proof}

What follows is a reformulation of the smoothness criteria for tensor fields (\cite[317--318]{lee:smooth_manifolds:2013}) for tensor fields of type $(1,k)$.

\begin{proposition}[Smoothness Criteria for Tensor Fields]
Let $M$ be smooth manifold and let $A : M \to T^{(1,k)}TM$ be a rough section. Then the following are equivalent:
\begin{enumerate}[label = \textup{(}\alph*\textup{)}]
\item $A \in \Gamma(T^{(1,k)}TM)$.
\item In every smooth coordinate chart, the component functions of $A$ are smooth.
\item For all $X_1,\dots,X_k \in \Xfrak(M)$, the rough section $A(X_1,\dots,X_k) : M \to TM$ defined by
\begin{equation}
A(X_1,\dots,X_k)(p) := A_p(X_1\vert_p,\dots,X_k\vert_p)
\end{equation}
\noindent is a smooth vector field.
\item If $X_1,\dots,X_k$ are smooth vector fields on some open subset $U \subseteq M$, then also $A(X_1,\dots,X_k)$ is a smooth vector field on $U$.
\end{enumerate}
\label{prop:smoothness_tensor}
\end{proposition}

\begin{proof}
We prove (a)$\Leftrightarrow$(b) and (b)$\Rightarrow$(c)$\Rightarrow$(d)$\Rightarrow$(b).\\
To prove (a)$\Leftrightarrow$(b), let $(U,(x^i))$ be a smooth chart. Actually, we can prove this for general tensor fields of type $(k,l)$. Proposition \ref{prop:smooth_bundle} together with the proof of the vector bundle chart lemma \cite[253--254]{lee:smooth_manifolds:2013} implies, that the corresponding chart on $T^{(k,l)}TM$ is given by $\del[1]{\pi^{-1}(U),\widetilde{\varphi}}$, where $\widetilde{\varphi}: \pi^{-1}(U) \to \varphi(U) \times \mathbb{R}^{n^{k+l}}$ is defined by
\begin{equation*}
\widetilde{\varphi} := (\varphi \times \id_{\mathbb{R}^{n^{k+l}}}) \circ \Phi
\end{equation*}
\noindent where $\Phi : \pi^{-1}(U) \to U \times \mathbb{R}^{n^{k + l}}$ is given as in the proof of proposition \ref{prop:smooth_bundle}. Now we consider the coordinate representation $\widehat{A}$ in the given charts (see \cite[35]{lee:smooth_manifolds:2013}). Since $A$ is a rough section, we have that 
\begin{equation*}
A^{-1}\del[1]{\pi^{-1}(U)} = (\pi \circ A)^{-1}(U) = \id_M^{-1}(U) = U.
\end{equation*}
Hence $\varphi\del[1]{U \cap A^{-1}(\pi^{-1}(U))} = \varphi(U)$, which is open, and $\widehat{A}: \varphi(U) \to \widetilde{\varphi}\del[1]{\pi^{-1}(U)}$ is given by
\begin{align*}
\widehat{A}(x) &= (\widetilde{\varphi} \circ A \circ \varphi^{-1})(x)\\
&= (\varphi \times \id_{\mathbb{R}^{n^{k+l}}})\del[1]{\Phi(\varphi^{-1}(x),A_{\varphi^{-1}(x)})}\\
&= (\varphi \times \id_{\mathbb{R}^{n^{k+l}}})\del[1]{\varphi^{-1}(x),\del[1]{A^{i_1\dots i_k}_{j_1\dots j_l}(\varphi^{-1}(x))}}\\
&= \del[1]{x,\del[1]{\widehat{A}^{i_1\dots i_k}_{j_1\dots j_l}(x)}}.
\end{align*}
By \cite[35]{lee:smooth_manifolds:2013} $A$ is smooth if and only if in any chart $\widehat{A}$ is smooth. This is furthermore equivalent to that each $\widehat{A}^{i_1\dots i_k}_{j_1\dots j_l}$ is smooth and thus equivalent to that $A^{i_1\dots i_k}_{j_1\dots j_l}$ is smooth (see \cite[33]{lee:smooth_manifolds:2013}).\\
To prove (b)$\Rightarrow$(c), let $(U,(x^i))$ be a smooth chart. Then write $X_1,\dots,X_k \in \Xfrak(M)$ as 
\begin{equation*}
X_\nu = X^{\mu_\nu}_\nu \frac{\partial}{\partial x^{\mu_\nu}}.
\end{equation*}
\noindent for $\nu = 1,\dots,k$. For $p \in U$ lemma \ref{lem:corr} implies
\begin{align*}
A(X_1,\dots,X_n)(p) &= A_p(X_1\vert_p,\dots,X_k\vert_p)\\
&= A_p\del[3]{X^{\mu_1}_1(p) \frac{\partial}{\partial x^{\mu_1}}\bigg\vert_p,\dots,X^{\mu_k}_1(p) \frac{\partial}{\partial x^{\mu_k}}\bigg\vert_p}\\
&= X^{\mu_1}_1(p) \cdots X^{\mu_k}_k(p)A_p\del[3]{\frac{\partial}{\partial x^{\mu_1}}\bigg\vert_p,\dots,\frac{\partial}{\partial x^{\mu_k}}\bigg\vert_p}\\
&= X^{\mu_1}_1(p) \cdots X^{\mu_k}_k(p) A^i_{\mu_1 \dots \mu_k}(p)\frac{\partial}{\partial x^i}\bigg\vert_p.
\end{align*}
By the smoothness criterion for vector fields \cite[175]{lee:smooth_manifolds:2013} we have that each component function $X^{\mu_n}_\nu$ is smooth. Thus if $A$ is smooth, we have by  that each $A^i_{j_1\dots j_k}$ is smooth and since $\Cscr^\infty(M)$ is an $\mathbb{R}$-algebra (see \cite[33]{lee:smooth_manifolds:2013}), we have that 
\begin{equation*}
X^{\mu_1}_1 \cdots X^{\mu_k}_k A^i_{\mu_1 \dots \mu_k}
\end{equation*} 
\noindent is smooth for $i = 1,\dots,n$. Thus again by the smoothness criterion together with the localness of smoothness \cite[35]{lee:smooth_manifolds:2013} we get that $A(X_1,\dots,X_k) \in \Xfrak(M)$.\\
To prove (c)$\Rightarrow$(d), we use that smoothness is a local property (see \cite[35]{lee:smooth_manifolds:2013}). Let $p \in U$.  Then by \cite[14]{cattaneo:manifolds:2017} we find a smooth bump function $\psi$ supported in $U$ and identically equal to $1$ on some neighbourhood $V$ of $p$. Set 
\begin{align*}
\widetilde{X}_i\vert_p := \begin{cases}
\psi(p) X_i\vert_p & p \in \supp \psi\\
0 & p \in M \setminus \supp \psi
\end{cases}.
\end{align*}
Then the gluing lemma for smooth maps \cite[35]{lee:smooth_manifolds:2013} implies $X_1,\dots,X_k \in \Xfrak(M)$. Hence by (c) we get that $A(\widetilde{X}_1,\dots,\widetilde{X}_k) \in \Xfrak(M)$ and so the restriction $A(\widetilde{X}_1,\dots,\widetilde{X}_k)\vert_V$ is smooth. But $A(\widetilde{X}_1,\dots,\widetilde{X}_k)\vert_V = A(X_1,\dots,X_k)$ and so we are done.\\
Lasty to prove (d)$\Rightarrow$(b), each vector field locally defined by 
\begin{equation*}
X_{j_\nu} = \delta^{\mu_\nu}_{j_\nu} \frac{\partial}{\partial x^{\mu_\nu}}.
\end{equation*}
\noindent is smooth. Thus by
\begin{equation*}
A(X_1,\dots,X_n)(p) = \delta^{\mu_1}_{j_1} \cdots \delta^{\mu_k}_{j_k} A^i_{\mu_1 \dots \mu_k}(p)\frac{\partial}{\partial x^i}\bigg\vert_p = A^i_{j_1\dots j_k}(p)\frac{\partial}{\partial x^i}\bigg\vert_p
\end{equation*}
\noindent we get that $A^i_{j_1 \dots j_k}$ is smooth and hence by (b) also $A$.
\end{proof}

\begin{theorem}[Tensor Characterization Lemma]
A mapping
\begin{equation*}
\underbrace{\mathfrak{X}(M) \times \dots \times \mathfrak{X}(M)}_{k} \to \Cscr^\infty(M) \qquad \text{or} \qquad \underbrace{\mathfrak{X}(M) \times \dots \times \mathfrak{X}(M)}_{k} \to \Xfrak(M)
\end{equation*}
\noindent is induced by an element of $\Gamma(T^{(0,k)}TM)$ or $\Gamma(T^{(1,k)}TM)$, respectively, if and only if they are multilinear over $\Cscr^\infty(M)$.
\label{thm:tensor_char}
\end{theorem}

\begin{proof}
We are proving only the second statement. Any element in $\Gamma(T^{(1,k)}TM)$ induces a mapping $\mathfrak{X}(M) \times \dots \times \mathfrak{X}(M) \to \Xfrak(M)$ by part (c) of the smoothness criteria for tensor fields \ref{prop:smoothness_tensor}. Thus we have to show that $\Ascr$ is multilinear over $\Cscr^\infty(M)$. Let $f \in \Cscr^\infty(M)$ and $X_\nu,\widetilde{X}_\nu \in \mathfrak{X}(M)$, $\nu = 1,\dots,k$. Then for any $p \in M$ we have that
\begin{align*}
\Ascr(X_1,\dots,fX_\nu + \widetilde{X}_\nu,\dots,X_k)_p =& A_p(X_1\vert_p,\dots,(fX_\nu + \widetilde{X}_\nu)_p,\dots,X_k\vert_p)\\
=& A_p(X_1\vert_p,\dots,f(p)X_\nu\vert_p + \widetilde{X}_\nu\vert_p,\dots,X_k\vert_p)\\
=& f(p)A_p(X_1\vert_p,\dots,X_\nu\vert_p ,\dots,X_k\vert_p)\\
& + A_p(X_1\vert_p,\dots,\widetilde{X}_\nu\vert_p,\dots,X_k\vert_p)\\
=& f(p) \Ascr(X_1,\dots,X_\nu,\dots,X_k)_p\\
& + \Ascr(X_1,\dots,\widetilde{X}_\nu,\dots,X_k)_p\\
=& \del[1]{f \Ascr(X_1,\dots,X_\nu,\dots,X_k)}_p \\
& + \Ascr(X_1,\dots,\widetilde{X}_\nu,\dots,X_k)_p.
\end{align*}
Conversly, suppose that $\Ascr: \mathfrak{X}(M) \times \dots \times \mathfrak{X}(M) \to \Xfrak(M)$ is multilinear over $\Cscr^\infty(M)$. Let $p \in M$. First we show that $\Ascr$ acts locally, i.e. if $X_\nu = \widetilde{X}_\nu$ in some neighbourhood $U$ of $p$ implies that also 
\begin{equation*}
\Ascr(X_1,\dots,X_\nu,\dots,X_k) = \Ascr(X_1,\dots,\widetilde{X}_\nu,\dots,X_k)
\end{equation*}
\noindent on $U$. By the multilinearity of $\Ascr$ it is enough to show that if $X_\nu$ vanishes on $U$ then so does $\Ascr$. There exists a smooth bump function $\psi$ for $\cbr[0]{p}$ supported in $U$ (see \cite[44]{lee:smooth_manifolds:2013}). Hence $\psi X_\nu = 0$ on $M$ and $\psi(p) = 1$. Thus
\begin{align*}
0 = \Ascr(X_1,\dots,\psi X_\nu,\dots,X_k)_p = \psi(p)\Ascr(X_1,\dots,X_\nu,\dots,X_k)_p.
\end{align*}
\noindent and since $\psi(p) = 1$ we have that
\begin{equation*}
\Ascr(X_1,\dots,X_\nu,\dots,X_k)_p = 0
\end{equation*}
\noindent for any $p \in U$.\\
Next we show that $\Ascr$ actually acts pointwise, i.e. if $X_\nu\vert_p$ vanishes so does $\Ascr$. Let $(U,(x^i))$ be a chart containing $p$ and $X_\nu = X_\nu^i \frac{\partial}{\partial x^i}$ on $U$. The same construction as used showing the implication (c)$\Rightarrow$(d) in the proof of proposition \ref{prop:smoothness_tensor} yields the existence of $f^1,\dots,f^n \in \Cscr^\infty(M)$ and $\widetilde{X}_1,\dots,\widetilde{X}_n \in \Xfrak(M)$ such that $f^i = X_\nu^i$ and $\widetilde{X}_i = \frac{\partial}{\partial x^i}$ on a neighbourhood $V \subseteq U$ of $p$. Thus by the previous localization, we get that 
\begin{equation*}
\Ascr(X_1,\dots,X_\nu,\dots,X_k) = \Ascr(X_1,\dots,f^i\widetilde{X}_i,\dots,X_k) = f^i\Ascr(X_1,\dots,\widetilde{X}_i,\dots,X_k) 
\end{equation*} 
\noindent in $U$. Since $0 = X_\nu^i(p) = f^i(p)$, $\Ascr$ vanishes at $p$. Hence $\Ascr$ depends only on the value of $X_\nu$ at $p$. Thus define a rough section $A : M \to T^{(1,k)}TM$ by 
\begin{equation*}
A_p(v_1,\dots,v_k) := \Ascr(V_1,\dots,V_k)(p)
\end{equation*}
\noindent where $V_1,\dots,V_k \in \Xfrak(M)$ are any extensions of $v_1,\dots,v_k \in T_pM$ (see \cite[177]{lee:smooth_manifolds:2013}). By the above, the choice of the extensions does not matter and the resulting rough section is smooth by proposition \ref{prop:smoothness_tensor} part (c), hence $A \in \Gamma(T^{(1,k)}TM)$.
\end{proof}

\end{appendix}

\printbibliography
\end{document}
