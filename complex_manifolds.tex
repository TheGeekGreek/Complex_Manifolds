\input{header.tex}
\newcommand{\Cscr}{\mathscr{C}}
\newcommand{\op}{\mathrm{op}}
\DeclareMathOperator\End{End}
\DeclareMathOperator\Hom{Hom}
\DeclareMathOperator\Mat{Mat}

\title{Complex Manifolds}
\author{Yannis B\"ahni}
\address[Yannis B\"ahni]{Universität Zürich, R\"amistrasse 71, 8006 Zurich}
\email[Yannis B\"ahni]{\href{mailto:yannis.baehni@uzh.ch}{yannis.baehni@uzh.ch}}

\begin{document}
\maketitle
\begin{abstract}
Goal of this paper is to give an overview of the basic definitions of complex and K\"ahler manifolds together with the most important properties. The main theorem will be the \emph{Newlander-Nirenberg Theorem} which gives a criterion under which an almost complex manifold is a complex one. The key role will be played by a certain tensor field, the so called \emph{Nijenhuis tensor}.
\end{abstract}
%\thispagestyle{fancy}

\tableofcontents

\section{Functions of Several Complex Variables}
This section summarizes the fundamental properties of functions of several complex variables needed later. The results are inspired by \cite[14--30]{grauert:complex_manifolds:2010}.

\begin{definition}
Let $n \in \mathbb{Z}$, $n \geq 1$, $U \subseteq \mathbb{C}^n$ open and $a \in U$. A mapping $f : U \to \mathbb{C}$ is said to be \bld{complex differentiable at $a$} if there exists $g : U \to \mathbb{C}^n$ such that $g$ is continuous at $a$ and
\begin{equation}
f(z) = f(a) + \sum_{\nu = 1}^n(z_\nu - a_\nu)g_\nu(z)
\end{equation} 
\noindent holds for all $z \in D$. $f$ is said to be \bld{holomorphic in $D$} if it is complex differentiable at every point $a \in D$. For $m \in \mathbb{Z}$, $m \geq 1$, a mapping $f : U \to \mathbb{C}^m$ is said to be holomorphic in $D$ if each component function $f_\nu$, $\nu = 1,\dots,n$, is holomorphic in $D$.
\label{def:holomorphic}
\end{definition}

\begin{proposition}
Let $n \in \mathbb{Z}$, $n \geq 1$, $D \subseteq \mathbb{C}^n$ open, $a \in U$ and $f : D \to \mathbb{C}$ real differentiable at $a$. Then
\begin{equation}
\frac{\partial f}{\partial z_\nu}(a) = \frac{1}{2}\del[3]{\frac{\partial f}{\partial x_\nu}(a) - i \frac{\partial f}{\partial y_\nu}(a)} 
\end{equation}
\noindent and
\begin{equation}
\frac{\partial f}{\partial \overline{z}_\nu}(a) = \frac{1}{2}\del[3]{\frac{\partial f}{\partial x_\nu}(a) + i \frac{\partial f}{\partial y_\nu}(a)}
\end{equation}  
\noindent holds for all $\nu = 1,\dots,n$.
\label{prop:Wirtinger}
\end{proposition}

\begin{theorem}[The Cauchy-Riemann Equations]
Let $n \in \mathbb{Z}$, $n \geq 1$ and $D \subseteq \mathbb{C}^n$ open. A mapping $f: D \to \mathbb{C}$ is holomorphic in $D$ if and only if it is real differentiable at every $a \in D$ and the \bld{Cauchy-Riemann equations}
\begin{equation}
\frac{\partial f}{\partial \overline{z}_\nu}(a) = 0 
\end{equation}
\noindent holds for all $a \in D$ and $\nu = 1,\dots,n$.
\label{thm:CR}
\end{theorem}

\begin{corollary}
Let $m,n \in \mathbb{Z}$, $m,n \geq 1$, $D \subseteq \mathbb{C}^n$ open and $f: D \to \mathbb{C}^m$ holomorphic in $D$. If $f = g + ih$, $g,h : D \to \mathbb{R}^m$, then 
\begin{equation}
\boxed{\frac{\partial g_\mu}{\partial x_\nu}(a) = \frac{\partial h_\mu}{\partial y_\nu}(a) \qquad \text{and} \qquad \frac{\partial h_\mu}{\partial x_\nu}(a) = -\frac{\partial g_\mu}{\partial y_\nu}(a)}
\end{equation} 
\noindent holds for any $a \in D$, $\nu = 1,\dots,n$ and $\mu = 1,\dots,m$.
\label{cor:CRreal}
\end{corollary}

\begin{proof}
Fix $\mu = 1,\dots,m$. By definition \ref{def:holomorphic} $f_\mu$ is holomorphic in $D$. Hence $f_\mu$ is real differentiable in $D$ (see \cite[27]{grauert:complex_manifolds:2010}) and theorem \ref{thm:CR} implies
\begin{equation*}
\frac{\partial f_\mu}{\partial \overline{z}_\nu}(a) = 0
\end{equation*}
\noindent for all $a \in D$ and $\nu = 1,\dots,n$. By proposition \ref{prop:Wirtinger}, this is equivalent to 
\begin{equation*}
\frac{\partial f}{\partial x_\nu}(a) + i \frac{\partial f}{\partial y_\nu}(a) = 0.
\end{equation*}
Using $f_\mu = g_\mu + i h_\mu$ and the $\mathbb{C}$-linearity of the operators $\frac{\partial}{\partial x_\nu}$ and $\frac{\partial}{\partial y_\nu}$ yields
\begin{equation*}
\frac{\partial g_\mu}{\partial x_\nu}(a) - \frac{\partial h_\mu}{\partial y_\nu}(a) + i \del[3]{ \frac{\partial h_\mu}{\partial x_\nu}(a) + \frac{\partial g_\mu}{\partial y_\nu}(a)} = 0.
\end{equation*}
\end{proof}

\section{Complex Structures on Vector Spaces}
In what follows, let $n \in \mathbb{Z}$, $n \geq 1$. Consider an $n$-dimensional complex vector space and let $J \in \End_\mathbb{C}(V)$ be defined by $J(v) := iv$. Then clearly $J \circ J = -\id_V$. Since every $n$-dimensional complex vector space can be seen as a $2n$-dimensional real vector space in a natural way, i.e. if $(e_\nu)$ is a basis for the complex vector space $V$, then $(e_\nu,ie_\nu)$ is a basis for the real vector space $V$, the mapping $J$ induces an $\mathbb{R}$-endomorphism $J$ on the real vector space $V$ simply by $J(e_\nu) = ie_\nu$ and $J(ie_\nu) = -e_\nu$ for all $\nu = 1,\dots,n$.\\
Conversly, let $V$ be an $n$-dimensional real vector space with $J \in \End_\mathbb{R}(V)$ such that $J \circ J = -\id_V$. One can show, that 
\begin{equation}
zv := xv + yJ(v)
\label{eq:complex}
\end{equation}
\noindent for $z := x + iy \in \mathbb{C}$ and $v \in V$ makes $V$ into a complex vector space. This motivates the following definition.
\begin{definition}
Let $V$ be an $n$-dimensional real vector space. A \bld{complex structure on $V$} is a $\mathbb{R}$-linear mapping $J: V \to V$ such that $J \circ J = -\id_V$. If $J$ is a complex structure on $V$, the tuple $(V,J)$ is called a \bld{complex vector space}.
\end{definition}

\begin{lemma}
Let $(V,J)$ be a complex vector space. Then $\dim V$ is even.
\label{lem:complex_vec}
\end{lemma}

\begin{proof}
That $\dim V$ must be even follows directly from
\begin{equation*}
\del[1]{\det(J)}^2 = \det(J \circ J) = \det(-\id_V) = (-1)^{\dim V} \det(\id_V) = (-1)^{\dim V}.
\end{equation*} 
\end{proof}

\section{Almost Complex Structures}
If $M$ is a smooth manifold, then $T_pM$ is a finite dimensional real vector space. Hence we can generalize the definitions and results of the previous section to manifolds. The following definition is taken from \cite[86]{cannas:symplectic_geometry:2008}.
\begin{definition}
Let $M$ be a smooth manifold. An \bld{almost complex structure on $M$} is a smooth tensor field $J \in \Gamma\del[1]{T^{(1,1)}TM}$ such that $J_p \circ J_p = -\id_{T_pM}$ holds for any $p \in M$. If $J$ is an almost complex structure on $M$, the tuple $(M,J)$ is called an \bld{almost complex manifold}.
\end{definition}

\begin{proposition}
Every almost complex manifold is of even dimension and orientable.
\label{prop:almost_comp}
\end{proposition}

\begin{proof}
Assume that $\dim M$ is odd. Let $p \in M$. Then by \cite[57]{lee:smooth_manifolds:2013} we have that $\dim T_p M = \dim M$. Hence $\dim T_pM$ is odd. But by lemma \ref{lem:complex_vec}, $\dim T_p M$ must be even since $(T_pM,J_p)$ is a complex vector space. Contradiction.\\
Assume that $J$ is an almost complex structure on $M$. Since $M$ is a smooth manifold, there exists a Riemannian metric $g$ on $M$ (see \cite[329]{lee:smooth_manifolds:2013}). Define 
\begin{equation*}
\widetilde{g}(X,Y) := g(X,Y) + g(JX,JY) \in \Gamma(T^{(0,2)}TM)
\end{equation*}
\noindent for all $X,Y \in \mathfrak{X}(M)$. Then
\begin{equation*}
\widetilde{g}(JX,JY) = g(JX,JY) + g(-X,-Y) = g(JX,JY) + g(X,Y) = \widetilde{g}(X,Y)
\end{equation*}
\noindent by the bilinearity of $g$. Furthermore, clearly $\widetilde{g}$ is positive definite and symmetric, thus a Riemannian metric on $M$. Define
\begin{equation*}
\omega(X,Y) := \widetilde{g}(X,JY).
\end{equation*}
Then by
\begin{equation*}
\omega(Y,X) = \widetilde{g}(Y,JX) = \widetilde{g}(JX,Y) = \widetilde{g}(-X,JY) = -\omega(X,Y)
\end{equation*}
\noindent we see that $\omega$ is skew-symmetric. Hence $\omega \in \Omega^2(M)$. Let $p \in M$ and $u \in T_pM \setminus \cbr{0}$. Then also $-J_p(u) \neq 0$ since $J_p$ is invertible since $\det J_p = 1$. Hence
\begin{equation*}
\omega\del[1]{u,-J_p(u)} = \widetilde{g}_p\del[1]{u,-(J_p \circ J_p)(u)} = \widetilde{g}(u,u) \neq 0
\end{equation*}
\noindent and by \cite[565]{lee:smooth_manifolds:2013} we get that $\omega$ is nondegenrate. Let $\dim M = 2n$. By \cite[567]{lee:smooth_manifolds:2013} this implies that $\omega_p \wedge \dots \wedge \omega_p$ is nonzero for each $p \in M$. Hence $\omega \wedge \dots \wedge \omega$ is a nonvanishing top form on $M$. Since any nonvanishing top form determines an orientation (see \cite[381]{lee:smooth_manifolds:2013}), we have that $M$ is orientable.
\end{proof}

\begin{remark}
The converse of proposition \ref{prop:almost_comp} is not true in general. One can show using results on fibre bundles and Chern classes, that if $\mathbb{S}^n$ admits an almost complex structure, then $n = 2^k - 2$ for $k \in \mathbb{Z}$, $k \geq 1$ (see \cite[219]{steenrod:topology_fibre:1951}). So for example $\mathbb{S}^4$ does not admit an almost complex structure. Actually, it can be shown that $\mathbb{S}^2$ and $\mathbb{S}^6$ are the only spheres which admit an almost complex structure (see \cite[434]{borel:steenrod:1953}).
\end{remark}

\section{Complex Manifolds}
The definition of smooth manifolds adapts smoothly to the complex case.

\begin{definition}
Let $n \in \mathbb{Z}$, $n \geq 1$. An \bld{$n$-dimensional complex manifold} is a second countable Hausdorff space $M$ equipped with a holomorphic structure, that is a holomorphic atlas $\cbr[0]{(U_\alpha, \varphi_\alpha) : \alpha \in A}$ of complex charts $(U_\alpha,\varphi_\alpha)$, such that all the transition maps are holomorphically compatible.
\end{definition}

\begin{lemma}
Let $V$ be a real vector space of dimension $n \in \mathbb{Z}$, $n \geq 1$. Then
\begin{equation}
V \otimes V^* \cong \End(V)
\end{equation}
\noindent canonically. If $(e_\nu)$ is a basis of $V$ and $(e_\nu^*)$ the corresponding basis of $V^*$, then $f \in \End(V)$ corresponds to
\begin{equation}
\sum_{\nu = 1}^n f(e_\nu) \otimes e_\nu^*.
\end{equation}
\label{lem:corr}
\end{lemma}

\begin{proof}
It is easily checked that
\begin{equation*}
\Phi:\begin{cases} 
V \times V^* \to \End(V)\\
(v,f) \mapsto \del[1]{u \mapsto f(u)v}
\end{cases}
\end{equation*}
\noindent is bilinear. Thus by the universal property of the tensor product there exists a unique mapping $\widehat{\Phi} \in \Hom(V \otimes V^*; \End(V))$ such that $\Phi = \widehat{\Phi} \circ \otimes$. It is also easily checked that $\widehat{\Phi}$ is an isomorphism.\\
Let $f \in \End(V)$. Then for any $v \in V$ we have
\begin{align*}
\widehat{\Phi}\del[3]{\sum_{\nu = 1}^n f(e_\nu) \otimes e_\nu^*}(v) &= \sum_{\nu = 1}^n \widehat{\Phi}\del[1]{f(e_\nu) \otimes e_\nu^*}(v)\\
&= \sum_{\nu = 1}^n e_\nu^*(v)f(e_\nu)\\
&= f\del[3]{\sum_{\nu = 1}^n e_\nu^*(v)e_\nu}\\
&= f(v).
\end{align*}
\end{proof}

\begin{proposition}
Any complex manifold admits a canonical almost complex structure.
\label{prop:complex_is_almost}
\end{proposition}

\begin{proof}
Fix a complex manifold $M$. We define $J$ in terms of local coordinates. Let $\del[1]{U,(x^\nu,y^\nu)}$ be a chart. By lemma \ref{lem:corr} it is also enough to construct an endomorphism $J_p$ for every $p \in U$. We define
\begin{equation*}
J_p\del[3]{\frac{\partial}{\partial x^\nu}\bigg\vert_p} := \frac{\partial}{\partial y^\nu}\bigg\vert_p \qquad \text{and} \qquad J_p\del[3]{\frac{\partial}{\partial y^\nu}\bigg\vert_p} := -\frac{\partial}{\partial x^\nu}\bigg\vert_p
\end{equation*}
\noindent for all $\nu = 1,\dots,n$. As standard linear algebra shows, there is a unique linear mapping associated with $J_p$ (see \cite[69]{hoffman:linear_algebra:1971}). Let $v := a^\nu\frac{\partial}{\partial x^\nu}\big\vert_p + b^\nu\frac{\partial}{\partial y^\nu}\big\vert_p \in T_p M$. Then
\begin{align*}
(J_p \circ J_p)(v) &=J_p\del[3]{a^\nu J_p\del[3]{\frac{\partial}{\partial x^\nu}\bigg\vert_p} + b^\nu J_p\del[3]{\frac{\partial}{\partial y^\nu}\bigg\vert_p}}\\
&= J_p\del[3]{a^\nu \frac{\partial}{\partial y^\nu}\bigg\vert_p - b^\nu \frac{\partial}{\partial x^\nu}\bigg\vert_p}\\
&= -a^\nu \frac{\partial}{\partial x^\nu}\bigg\vert_p - b^\nu \frac{\partial}{\partial y^\nu}\bigg\vert_p\\
&= -v
\end{align*}
\noindent and thus $J_p \circ J_p = -\id_{T_pM}$.\\
Next we have to show that above locally defined mapping is well-defined, i.e. does not depend on the choice of coordinates. Assume that $p \in U \cap V$ for another chart $\del[1]{V,(u^i,v^i)}$. By the change of coordinates formula \cite[64]{lee:smooth_manifolds:2013} we get that
\begin{equation*}
\frac{\partial}{\partial x^\nu}\bigg\vert_p = \frac{\partial u^\mu}{\partial x^\nu}(\widehat{p})\frac{\partial}{\partial u^\mu}\bigg\vert_p + \frac{\partial v^\mu}{\partial x^\nu}(\widehat{p})\frac{\partial}{\partial v^\mu}\bigg\vert_p
\end{equation*}
\noindent and
\begin{equation*}
\frac{\partial}{\partial y^\nu}\bigg\vert_p = \frac{\partial u^\mu}{\partial y^\nu}(\widehat{p})\frac{\partial}{\partial u^\mu}\bigg\vert_p + \frac{\partial v^\mu}{\partial y^\nu}(\widehat{p})\frac{\partial}{\partial v^\mu}\bigg\vert_p 
\end{equation*}
\noindent where $\widehat{p}$ denotes the coordinate representation of $p$ with respect to the coordinates $(x^\nu,y^\nu)$. Corollary \ref{cor:CRreal} implies
\begin{align*}
J_p\del[3]{\frac{\partial}{\partial x^\nu}\bigg\vert_p} &= \frac{\partial u^\mu}{\partial x^\nu}(\widehat{p})J_p\del[3]{\frac{\partial}{\partial u^\mu}\bigg\vert_p} + \frac{\partial v^\mu}{\partial x^\nu}(\widehat{p})J_p\del[3]{\frac{\partial}{\partial v^\mu}\bigg\vert_p}\\
&= \frac{\partial u^\mu}{\partial x^\nu}(\widehat{p})\frac{\partial}{\partial v^\mu}\bigg\vert_p - \frac{\partial v^\mu}{\partial x^\nu}(\widehat{p})\frac{\partial}{\partial u^\mu}\bigg\vert_p\\
&= \frac{\partial v^\mu}{\partial y^\nu}(\widehat{p})\frac{\partial}{\partial v^\mu}\bigg\vert_p + \frac{\partial u^\mu}{\partial y^\nu}(\widehat{p})\frac{\partial}{\partial u^\mu}\bigg\vert_p\\
&= \frac{\partial}{\partial y^\nu}\bigg\vert_p
\end{align*}
\noindent and
\begin{align*}
J_p\del[3]{\frac{\partial}{\partial y^\nu}\bigg\vert_p} &= \frac{\partial u^\mu}{\partial y^\nu}(\widehat{p})J_p\del[3]{\frac{\partial}{\partial u^\mu}\bigg\vert_p} + \frac{\partial v^\mu}{\partial y^\nu}(\widehat{p})J_p\del[3]{\frac{\partial}{\partial v^\mu}\bigg\vert_p}\\
&= \frac{\partial u^\mu}{\partial y^\nu}(\widehat{p})\frac{\partial}{\partial v^\mu}\bigg\vert_p - \frac{\partial v^\mu}{\partial y^\nu}(\widehat{p})\frac{\partial}{\partial u^\mu}\bigg\vert_p\\
&= -\frac{\partial v^\mu}{\partial x^\nu}(\widehat{p})\frac{\partial}{\partial v^\mu}\bigg\vert_p - \frac{\partial u^\mu}{\partial x^\nu}(\widehat{p})\frac{\partial}{\partial u^\mu}\bigg\vert_p\\
&= -\frac{\partial}{\partial x^\nu}\bigg\vert_p.
\end{align*}
Left to check is smoothness. According to lemma \ref{lem:corr} the corresponding rough tensor field is given by
\begin{equation*}
J_p\del[3]{\frac{\partial}{\partial x^\nu}\bigg\vert_p} \otimes \d x^\nu\vert_p + J_p\del[3]{\frac{\partial}{\partial y^\nu}\bigg\vert_p} \otimes \d y^\nu\vert_p = \frac{\partial}{\partial y^\nu}\bigg\vert_p \otimes \d x^\nu\vert_p - \frac{\partial}{\partial x^\nu}\bigg\vert_p \otimes \d y^\nu\vert_p
\end{equation*}
\noindent for any $p \in U$. Thus the smoothness criteria for tensor fields \cite[317--318]{lee:smooth_manifolds:2013} together with \cite[36]{lee:smooth_manifolds:2013} yields that $J \in \Gamma\del[1]{T^{(1,1)}TM}$.
\end{proof}

A question which naturally arises by considering proposition \ref{prop:complex_is_almost} is, if the converse is also true, i.e. if every almost complex manifold is a complex manifold. This is in general not the case. Let $\mathbb{P}$ denote the naturally oriented underlying smooth manifold of the complex projective plane. Again using results about Chern numbers it can be shown that
\begin{equation}
\mathbb{P} \# (\mathbb{S}^1 \times \mathbb{S}^3) \# (\mathbb{S}^1 \times \mathbb{S}^3) \qquad \text{and} \qquad (\mathbb{S}^1 \times \mathbb{S}^3) \# (\mathbb{S}^1 \times \mathbb{S}^3) \# (\mathbb{S}^2 \times \mathbb{S}^2)
\end{equation}
\noindent have almost complex structures but no complex structure (see \cite[1627]{ven:chern:1966}). 

\section{The Nijenhuis Tensor and the Newlander-Nirenberg Theorem}
As we have seen in the last section, not every almost complex manifold is a complex manifold. Under which condition is this possible?

\begin{definition}
Let $(M,J)$ be an almost complex manifold. For $X,Y \in \mathfrak{X}(M)$ we define the \bld{Nijenhuis tensor $N$} as
\begin{equation}
N(X,Y) := \sbr[0]{JX,JY} - J\sbr[0]{X,JY} - J\sbr[0]{JX,Y} - \sbr[0]{X,Y}
\end{equation}
\noindent where $\sbr[0]{X,Y}$ denotes the usual Lie-bracket of vector fields.
\end{definition}

\begin{proposition}
Let $(M,J)$ be an almost complex manifold and $N$ be the associated Nijenhuis tensor. Then $N \in \Gamma\del[1]{T^{(1,2)}TM}$.
\label{prop:nijenhuis_tensor}
\end{proposition}

\begin{proof}
First of all, $N(X,Y) \in \mathfrak{X}(M)$ for all $X,Y \in \mathfrak{X}(M)$. This follows immediately by considering $J$ as a mapping $J : \mathfrak{X}(M) \to \mathfrak{X}(M)$ (see \cite[26]{kobayashi:diff_geo_I:1996}), the fact that the Lie Bracket of two smooth vector fields is again a smooth vector field (see \cite[186]{lee:smooth_manifolds:2013}) and that $\mathfrak{X}(M)$ is a $\Cscr^\infty(M)$-module (see \cite[177]{lee:smooth_manifolds:2013}). Let $f \in \Cscr^\infty(M)$ and $X,Y,Z \in \mathfrak{X}(M)$. Then
\begin{align*}
N(fX + Y,Z) =& \sbr[0]{J(fX + Y),JZ} - J\sbr[0]{fX + Y,JZ} - J\sbr[0]{J(fX + Y),Z}\\
 &- \sbr[0]{fX + Y,Z}\\
=& \sbr[0]{fJX + JY,JZ} - J\sbr[0]{fX + Y,JZ} - J\sbr[0]{fJX + JY,Z}\\
& - \sbr[0]{fX + Y,Z}\\
=& \sbr[0]{fJX,JZ} + \sbr[0]{JY,JZ} - J\sbr[0]{fX,JZ} - J\sbr[0]{Y,JZ} - J\sbr[0]{fJX,Z}\\
& -J\sbr[0]{JY,Z} - \sbr[0]{fX,Z} - \sbr[0]{Y,Z}\\
=& f\sbr[0]{JX,JZ} - (JZf)JX + \sbr[0]{JY,JZ} - fJ\sbr[0]{X,JZ} + (JZf)JX\\
& - \sbr[0]{Y,JZ} - fJ\sbr[0]{JX,Z} + (Zf)JJX - J\sbr[0]{JY,Z} - f\sbr[0]{X,Z}\\
& + (Zf)X - \sbr[0]{Y,Z}\\
=& fN(X,Z) + N(Y,Z).
\end{align*}
\noindent by \cite[187--188]{lee:smooth_manifolds:2013}. Linearity in the second argument is shown similarly. Hence $N : \mathfrak{X}(M) \times \mathfrak{X}(M) \to \mathfrak{X}(M)$ is bilinear over $\Cscr^\infty(M)$. So by \cite[26]{kobayashi:diff_geo_I:1996} we have that $N \in \Gamma\del[1]{T^{(1,2)}TM}$.
\end{proof}

\begin{theorem}[Newlander-Nirenberg]
Let $(M,J)$ be an almost complex manifold. Then $M$ is a complex manifold, where the complex structure is so that the canonically induced almost complex structure is $J$, if and only if the Nijenhuis tensor $N$ vanishes identically.
\end{theorem}

\begin{proof}
Assume $M$ is a complex manifold. Let $(U,(x^\nu,y^\nu))$ be a chart. From proposition \ref{prop:nijenhuis_tensor} it is enough to consider the coordinate vector fields $\frac{\partial}{\partial x^\nu}$ and $\frac{\partial}{\partial y^\nu}$. But from the explicit definition of $J$  in proposition \ref{prop:complex_is_almost} and the property, that the Lie-Bracket of coordinate vector fields vanishes, together with the $\Cscr^\infty(M)$-linearity of $J$ we get that $N$ vanishes identically on each chart, and thus on $M$.\\
The other direction however is far more technical and uses results on partial differential equations. A complete proof can either be found in the original paper \cite{newlander:coordinates:1957} or in \cite[106]{cannas:symplectic_geometry:2008}, where references to more recent proofs are given.
\end{proof}

\section{K\"ahler Manifolds}

\printbibliography
\end{document}
